\documentclass[a4paper,10pt,aip,onecolumn,amsmath,amssymb,floatfix,rmp]{revtex4-1}
%\documentclass[book,amsmath,amssymb,floatfix]{revtex4-1}
%\documentclass[preprint,showpacs,preprintnumbers,amsmath,amssymb]{revtex4-1}

% Some other (several out of many) possibilities
%\documentclass[aps,prl,preprint,groupedaddress,showpacs]{revtex4}
%\documentclass[aps,prl,twocolumn,superscriptaddress,showpacs]{revtex4}
%\documentclass[aps,prb,preprint,superscriptaddress,showpacs]{revtex4}

\usepackage{doi}
\usepackage{graphicx}% Include figure files
\usepackage{longtable}
\usepackage{dcolumn}% Align table columns on decimal point
\usepackage{bm}% bold math
%\usepackage{pst-all}      % From PSTricks
\usepackage{fancybox}
%\topmargin=.0cm
%\nofiles

\newcommand{\rou}[1]{\noindent------------------------------------------------------------------------------------------------------------
\noindent{\bf \large #1}}
\newcommand{\fl}[1]{\noindent{\sf $\bullet$ #1\index{\sf #1}} : }
\newcommand{\fx}[1]{\subsection{\sf #1\index{\sf #1}}}
\newcommand{\ssx}[1]{\subsection{\bf #1\index{\bf #1}}}
\newcommand{\ssxx}[2]{\subsection{\bf #1\index{\bf #2}}}
\newcommand{\infiles}{\noindent\fbox{Input files}}
\newcommand{\outfiles}{\noindent\fbox{Output files}}
\newcommand{\GW}{$GW$}
\newcommand{\GWinput}{{\sf GWinput}\ }
\newcommand{\GWIN}{{\sf GWIN}\ }

\newcommand{\gbox}[1]{\noindent{\color{Green}\fbox{\parbox{260mm}{#1}}}}
\newcommand{\rbox}[1]{\noindent{\color{Red}\fbox{\parbox{260mm}{#1}}}}
\newcommand{\obox}[1]{\noindent{\color{Orange}\fbox{\parbox{260mm}{#1}}}}
\newcommand{\cyanbox}[1]{\noindent{\color{Cyan}\fbox{\parbox{260mm}{#1}}}}
\newcommand{\bluebox}[1]{\noindent{\color{Blue}\fbox{\parbox{260mm}{#1}}}}

\newcommand{\keyw}[1]{\fbox{\tt #1}}
\newcommand{\innera}[1]{{[\![#1]\!]_{R_a}}}
\newcommand{\bfq}{{\bf q}}
\newcommand{\bfk}{{\bf k}}
\newcommand{\bfr}{{\bf r}}
\newcommand{\hbfr}{\hat{\bf r}}
\newcommand{\bfQ}{{\bf Q}}
\newcommand{\bfT}{{\bf T}}
\newcommand{\bfG}{{\bf G}}
\newcommand{\bfR}{{\bf R}}
\newcommand{\ds}{\displaystyle}

\newcommand{\exe}[1]{{\bf #1}}
\newcommand{\io}[1]{{\sf  #1}}
\newcommand{\raw}[1]{{\tt #1}}
\newcommand{\repp}[1]{p.\pageref{#1}}

\newcommand{\refeq}[1]{Eq.~(\ref{#1})}
\newcommand{\reffig}[1]{Fig.\ref{#1}}

\newcommand{\commentout}[1]{\ \\ \noindent xxxxxxxxxx comment out. from
here xxxxxxxxxxxxxxx\\ #1 \\ xxxxxxxxxx to here xxxxxxxxxxxxxxxxxxxxxxxxxxxxxx\\}

\newcommand{\smH}{{\mathcal H}}
\newcommand{\YY}{{\cal Y}}
\newcommand{\GG}{{\cal G}}

\newcounter{Alist}
\newcommand{\ul}[1]{\underline{#1}}
%%%%%%%%%%%%%%%%%%%%%%%%%%%%%%%%
\newcommand{\ocite}[1]{\cite{#1}}


\newcommand{\ispone}{}
\newcommand{\isptwo}{}
\newcommand{\ooplus}{\oplus}
\newcommand{\oominus}{\ominus}

%\def\psibar{\bar{\psi}}
%\def\psidotbar{\dot{\bar{\psi}}}
%\def\scgw{{sc{\em GW}}}
\def\tphi{{\tilde{\phi}}}
\def\calR{{\cal A}}
\def\qsgw{QS{\em GW}}
\def\ldagw{{lda{\em GW}}}
\def\GLDA{{G^{\rm LDA}}}
\def\WLDA{{W^{\rm LDA}}}
\def\ekn{{\varepsilon_{{\bf k}n}}}
\def\phidot{\dot{\phi}}
\def\phidottilde{\dot{\tilde{\phi}}}
\def\phitilde{\tilde{{\phi}}}
\def\epsilonaone{\epsilon^{(1)}_a}
\def\epsilonatwo{\epsilon^{(2)}_a}
\def\ei{\varepsilon_i}
\def\eis{\varepsilon_{i\sigma}}
\def\ej{\varepsilon_j}
\def\Ekn{{E_{{\bf k}n}}}
\def\Psikn{\Psi_{{\bf k}n}}
\def\Psiqn{{\Psi_{{\bf q}n}}}
\def\Psiqm{{\Psi_{{\bf q}m}}}
\def\DVo{{\it \Delta}V(\omega)}
\def\DVoret{{\it \Delta}V^R(\omega)}
\def\DVoadv{{\it \Delta}V^A(\omega)}
\def\DV{{\it \Delta}V}
\def\DVhat{{\it \Delta}\hat{V}}
\def\HLDA{H^{\rm LDA}}
\def\H0{H^0}
\def\hH{\hat{H}}
\def\veff{V^{\rm eff}}
\def\vxc{V^{\rm xc}}
\def\vc{V^{\rm c}}
\def\vext{V^{\rm ext}}
\def\hVext{\hat{V}^{\rm ext}}
\def\hVeff{\hat{V}^{\rm eff}}
\def\hvnl{\hat{V}^{\rm nl}}
\def\vnl{V^{\rm nl}}
\def\vh{V^{\rm H}}
\def\vgw{V^{GW}}
\def\ReDVo{ {\rm Re}[{\it \Delta}V(\omega)] }
\def\ImDVo{ {\rm Im}[{\it \Delta}V(\omega)] }
\def\gwa{$GW$\!A}
\def\hVee{\hat{V}^{\rm ee}}
\def\hVext{\hat{V}^{\rm ext}}
\def\hHk{\hat{H}^{\rm k}}
\def\Sigmax{{\Sigma}^{\rm x}}
\def\Sigmac{{\Sigma}^{\rm c}}
\newcommand{\req}[1]{\mbox{Eq.~\!(\ref{#1})}}
\newcommand{\refsec}[1]{\mbox{Sec.~\!\ref{#1}}}
\def\Heff{\hat{H}^{\rm eff}}
\def\vbar{\bar{V}}
\newcommand{\hHzero}{\hat{H}^{0}}


\def\Sbarc{\bar{\Sigma}^{\rm c}}

\def\scgw{{QS{\em GW}}}
\def\ldagw{{lda{\em GW}}}
%\def\GLDA{{G^{\rm LDA}}}
%\def\WLDA{{W^{\rm LDA}}}
\def\ekn{{\varepsilon_{{\bf k}n}}}
\def\ekm{{\varepsilon_{{\bf k}m}}}
\def\eknp{{\varepsilon_{{\bf k}n'}}}
\def\Ekn{{E_{{\bf k}n}}}
\def\Psikn{{\Psi_{{\bf k}n}}}
\def\Psikm{{\Psi_{{\bf k}m}}}
\def\Psikmstar{{ \Psi_{{\bf k}m}^*} }
\def\Psiknp{{\Psi_{{\bf k}n'}}}
\def\Psikqn{{\Psi_{{\bf k}+{\bf q} n}}}
\def\Psikm{{\Psi_{{\bf k}m}}}
\def\Psikmstar{{ \Psi_{{\bf k}m}^*} }
\def\Psiknp{{\Psi_{{\bf k}n'}}}

%\def\rs{r_{\rm s}}

\newcommand{\CikL}{{C^{(i)}_{kL}}}
\newcommand{\CiRkL}{{C^{(i)}_{{\bf R}kL}}}
\newcommand{\tPkL}{{\widetilde{P}_{kL}}}
\newcommand{\tPRkL}{{\widetilde{P}_{{\bf R}kL}}}
\newcommand{\PkL}{{P_{kL}}}
\newcommand{\PRkL}{{P_{{\bf R}kL}}}
%\newcommand{\rmt}{{r_{MT}}}
\newcommand{\rmt}{{s_{\bf R}}}

\def\brl{{\bf R}L}
\def\brlp{{{\bf R}'L'}}
%\def\brl{{\bf a}L}
%\def\brlp{{{\bf a}'L'}}

\def\tili{{\widetilde{i}}}
\def\tilj{{\widetilde{j}}}
\def\tiln{{\widetilde{n}}}
\def\tilm{{\widetilde{m}}}
\newcommand{\val}{{\rm{VAL}}}
\newcommand{\core}{{\rm{CORE}}}
\newcommand{\xc}{_{\rm{xc}}}

\newcommand{\CORE}{{CORE}}
\newcommand{\COREone}{{CORE1}}
\newcommand{\COREtwo}{{CORE2}}
\newcommand{\VAL}{{\rm{VAL}}}
\newcommand{\EF}{E_{\rm F}}
\newcommand{\oneshotgw}{1shot-$GW$}
\def\tn{\tilde{n}}
\def\tnT{\tilde{n}^{\rm T}}
\def\tV{\tilde{V}}
\def\tnT{\tilde{n}^{\rm T}}


\def\eak{\varepsilon_{\rm a}(\bfk)}
\def\ebk{\varepsilon_{\rm b}(\bfk)}
\def\iDelta{{\it \Delta}}
\def\efermi{\mbox{$E_{\rm F}$}}

\def\connect#1{\leavevmode{\setbox1=\hbox{#1}\copy1%
\raise .2\ht1 \vbox{\moveleft \wd1\vbox{\hrule width \wd1 height .5pt depth 0pt}}%
}}
%\def\we{\connect{\mbox{$\omega\varepsilon$}}}
\def\we{\mbox{$\omega_\varepsilon$}}
\def\eal{\varepsilon_{al}}
\def\eallo{\varepsilon^{\rm Lo}_{al}}
\def\smh{smHankel}
\def\smhs{smHankels}
\def\shotone{OneShot}
\def\shotonez{OneShot Z=1}
\def\x{\mbox{$\times$}}

\def\xccut{ {\rm xccut} }
\def\xccutone{ {\rm xccut1} }
\def\xccuttwo{ {\rm xccut2} }


\def\ftn[#1]{\rlap{\footnotemark[#1]}}

\def\tr{{\rm Tr}}

\def\bQP{{\it bare QP}}
\def\bQPs{{\it bare QPs}}
\def\dQP{{\it dressed QP}}
\def\dQPs{{\it dressed QPs}}
\def\Re{{\rm Re}}

%\def\pmdelta{\pm i\delta}
%\def\pmdelta{i \mbox{\cal sign}(\efermi -\ei) \delta}

\def\EMAX{  E^{\rm APW}_{\rm MAX} }
\def\EMAXm{ E^{\rm rmesh}_{\rm MAX} }
\def\NAPW{N_{\rm APW}}
\def\RSM{R_{\rm SM}}
\def\RSMa{R_{{\rm SM},a}}
\def\RSMal{R_{{\rm SM},al}}
\def\epsilonal{\epsilon_{al}}
\def\RGS{R_{\rm G}}
\def\RGSa{R_{{\rm G},a}}
\def\pakl{p_{akl}}
\def\PakL{P_{akL}}
\def\wPakL{\widetilde{P}_{akL}}
\def\CakL{C_{akL}}
\def\CiakL{C^i_{akL}}

\def\EMAX{  E^{\rm APW}_{\rm MAX} }
\def\EMAXm{ E^{\rm rmesh}_{\rm MAX} }

\def\nc{n^{\rm c}}
\def\nzc{n^{\rm Zc}}
\def\nzcv{n^{\rm Zcv}}
\def\barnzcv{\bar{n}^{\rm Zcv}}
\def\MM{{\cal M}}
\def\RR{v}
\def\inta{\int_{|\bfr|\leq R_a}\!\!\!\!\!\!\!\!\!\!\!\!}
\def\intaa{\int_{|\bfr|\leq R_a}}
\def\intad{\int_{|\bfr'|\leq R_a}\!\!\!\!\!\!\!\!\!\!\!\!}
\def\intar{\int_{|\bfr-\bfR_a|\leq R_a}}
\def\intard{\int_{|\bfr'-\bfR_a|\leq R_a}}
\def\rhoij{\rho_{ij}}
\def\ekcore{E_{\rm k}^{\rm core}}
\def\ek{E_{\rm k}}
\def\ehf{E_{\rm Harris}}
\def\nin{n^{\rm in}}
\def\nout{n^{\rm out}}
%\def\Vin{V^{\rm in}}
\def\Vin{V}
\def\iDelta{{\it \Delta}}
\def\philo{{\phi}^{\rm Lo}_{al}}
\def\DEe{{\it \Delta} E_{\rm e}}


%\bibliographystyle{alpha}
\setlength{\topmargin}{-10mm}
\setlength{\textwidth}{140mm}
\setlength{\textheight}{240mm}

%%%%%%%%%%%%%%%%%%%%%%%%%%%%%%%%%%%%%%%%%%%%%%%%%%%%%%%%%%%%%%%%%%%%%%%%%%
\begin{document}
\special{papersize=210 mm, 297mm}
\title{{\Huge \ \\ ecalj package note:}\\ the PMT method and the
PMT-QSGW method  @ https://github.com/tkotani/ecalj}
\author{\large \ \\ Takao Kotani}
\affiliation{Department of applied mathematics and physics,\\ Tottori university, Tottori 680-8552, Japan}
\author{\large Hiori Kino}
\affiliation{National Institute for materials science, Sengen 1-2-1, Tsukuba, Ibaraki 305-0047, Japan.}
% \author{Hisazumu Akai}
% %\affiliation{Department of Physics, Osaka University, Toyonaka, Osaka 560-0043, Japan}
\date{\today}
\vspace{1cm}
\begin{abstract}
\ \\ The PMT method, which is an all-electron full-potential (FP) mixed-basis
method for the first-principle electronic structure calculation, was proposed 
by the author in Phys.Rev.B81 125117(2010). 
It is a hybrid of the FP-LAPW and FP-LMTO, that is, a method 
whose basis set consists of both of the augmented plane waves (APW) 
and the muffin-tin orbitals (MTO). We give improvements for the PMT
method. 

 As we show elsewhere, it works fine even for diatomoic molecules
from H2 to Kr2; we can obtain atomization energy within chemical
accuracy with the need cutoff energy of APWs about 3 to 4 Ry. This is
because we inclue MTOs.

Then we explain how to implement the quasi-particle
self-conisistent GW (QSGW) method in the PMT method.
A key point is in how to repsesent static version of the self-energy 
in the IBZ.\\

What we can do now:
\begin{itemize}
\item GGA/LDA calculations, atomic force(relaxation and dynamics), total
      energy, band, dos. LDA+U. Spin-orbit coupling LdotS (but
      non-colinear yet). k-point paralell.
\item one-shot GW, quasiparticle self-consistent GW. dielectric function. Spin
      fluctuation. MPI.
\item (not implemented well, but it may work). Maxloc Wannier function.
      effective coulomb interaction in RPA.
\end{itemize}

git管理してるのでバージョンはgitで確認のこと。

この文書は、おもにpmttheory.tex(未発表:kotani,kino,and akai)に解説した理論をもとに、
PMT法(MTOとAPWによる混合基底の方法)とPMT-QSGW法(PMT法をもとにしたQSGW
法、さらには電荷ゆらぎなど)がいかに実装されているかを記述するものである。\\

英語による記述と日本語による記述が混在している。
またメモなども混在している。方針は、
「英語部分だけでもそれなりに意味をなす。
日本語部分は多少重複するかもしれないが混乱を招かない範囲で、
理解しやすさのために残していく。
新しい情報は日本語ででも書く(必要に応じて英語になおす)。」
と言う方針をとることとする。
この文書は、徐々に更新、改訂を加えていくつもりである。
\end{abstract}
%\pacs{71.15.Ap, 71.15.Fv 71.15.-m}
\maketitle
\tableofcontents

% --------------- Introduction ------------------
\section{introduction}
``ecalj package''は、おもにはPMT法およびそれを元にしたQSGW法(PMT-QSGW法) 
を実行するためのコードである。PMT法はAPWとMTOを同時に使う
混合基底の第一原理計算の方法である。周期境界条件を用いる。
GGA(PBE)のレベルで原子に働く力を求めての構造緩和なども可能である.
一体問題を解く部分をlmf,QSGWを解く部分をfpgwと呼ぶ。\\

In the first-principle electronic structure calculations based on the
density functional theory in the LDA/GGA
(local density approximation/generalized gradient approximation),
a key element is the one-body problem solver, which 
should have efficiency, accuracy and robustness.
%Even in methods to go beyond DFT such as the quasiparticle
%self-consistent $GW$ (QSGW) \cite{kotani07a}, the solver is a key
%to obtain accurate results with minimum computational efforts.
As such solvers, the linearized augmented plane wave (LAPW) method and the 
linearized muffin-tin orbital (LMTO) method were proposed by Andersen
in 1975 \cite{Andersen75}, followed by many improvements
and extensions \cite{rmartinbook,Singhbook,bluegel31,lmfchap,PAW,PhysRevB.43.6388}.
Today LAPW and LMTO has developed to be full-potential methods, 
which we treat in this paper.
In these methods, wavefunctions are represented by superpositions of
augmented waves used as a basis. The LAPW uses the augmented plane waves (APWs) made 
of plane waves (PWs) as envelope functions. 
The LMTO uses the muffin-tin orbitals (MTOs) made of the atom-centered 
Hankel functions. Corresponding to these envelope functions, the APWs fit to
the extended nature of wavefunctions; 
in contrast, the MTOs to the localized nature of them.
However, wavefunctions in real materials should have both the natures.

This fact is reflected as shortcomings in these methods.
In the case of the LAPW, it requires too many bases 
in order to represent sharp structures of wavefunctions 
just outside of muffin-tins. For example, 3$d$ orbitals of transition
metals are the typical cases. Most of all the PWs used in the LAPW method 
are consumed only to reproduce the sharp structures.
On the other hand, the LMTO is problematic in representing 
the extended nature of wavefunctions. For example, we sometimes need to 
put empty spheres between atoms. In addition, it is not 
simple to enlarge basis set systematically 
in order to check numerical convergence.

To overcome these shortcomings, 
Kotani and van Schilfgaarde introduced a new linearized method 
named as the APW and MTO method (the PMT method) \cite{pmt1}, which is an all-electron (AE) 
full-potential mixed basis method using APWs and MTOs simultaneously.
Because these two kinds of basis are complementary, 
we can overcome these shortcomings. Within our knowledge, 
no other mixed basis methods have used different kinds 
of augmented waves simultaneously in the full-potential methods.

A minimum description on the formalism of the PMT method is given in Ref.\cite{pmt1}, 
which is based on Ref.\onlinecite{lmfchap} for a LMTO method  
by Methfessel et al. However, the formalism was not very transparent, 
mainly because it was not derived from the explicit total energy minimization.
This makes theoretical treatment of the PMT method somehow complicated.
For example, it resulted in a complicated logic to derive atomic forces 
in Refs.\onlinecite{molforce,lmfchap}. 
%Since we are afraid of
%the Pulay terms \cite{pulay69}, it should be derived from the derivative
%of total energy. 
It was not easy to compare the PMT method with the projector augmented
wave (PAW) methods \cite{PAW,kresse99} on the same footing.
Thus we should give a simple and clear formalism to the PMT method 
for its further developments rather than that in Refs.\onlinecite{pmt1,lmfchap}.

In this paper, we introduce a formalism, named as the 3-component
formalism, which is a mathematically transparent generalization of the additive
augmentation given by Soler and Williams \cite{soler89,soler90,soler93} 
(See discussion in Sec.VII in Ref.\cite{PAW}).
We give a formalism of the PMT method based on the 3-component formalism.
In the PAW method \cite{PAW}, the total energy is minimized as a
functional of pseudo wavefunctions. In the 3-component formalism,
the minimization is formulated as for the wavefunctions represented in the 
{\it 3-component space} defined in \refsec{sec:formalism} under some constraints.
This is somehow general in the sense that it allows to use any kinds of basis 
(need not to be given by projectors); thus it is
suitable to formulate mixed basis methods such as the PMT.
Results of the PMT method applied to diatomic molecules from H$_2$ through Kr$_2$ 
are given in elsewhere. Considering the fact
that the PMT method (even the LMTO itself) is already pretty good to
describe solids \cite{pmt1,lmfchap,kotani07a,mark06adeq},
the PMT method can be a candidate to perform full-potential
calculations for molecules and solids in an unified manner, 
more efficiently than LAPW.

In Sec.\ref{sec:formalism}, we will give the 3-component formalism.
Functional relations of physical quantities become transparent in the formalism.
In Sec.\ref{sec:pmtmethod}, we give the formulation of the PMT method
based on the 3-component formalism. Then we discuss problems
in the PMT method and ways to overcome them, giving a comparison with the PAW method.
Derivation of atomic forces becomes straightforward as given in Appendix
without any confusion that were discussed in Ref.\onlinecite{soler93}.\\


以下では、PMT法を``the 3-component augmentation for bilinear product''と名づけた考え方
によって、解説する。この考え方の軸は、augmentationを基底関数そのものにではなく、その積に
適用することである。
%As the name stands for, the augmentation is applied not to the
%envelope function themselves, but to the products of them. 
%After the explanation of the \smh\ functions in \refsec{shankel},
%we explain the one-center expansion within MTs in
%\refsec{sec:onecenter}.

\subsection{PMT法の基本の考え方}
PMT法の母体はFP-LMTOのパッケージ(lmf package)から進化させたものである。
その母体であるlmfの主文献は二つあり、\cite{lmfchap}と\cite{nfpmanual}で
ある。これらは、/MarksOriginalDoc/nfp-doc.ps.gzと
/MarksOriginalDoc/nfp-manual.ps.gz
に格納されている。


PMT法の基本アイデアを記述した文献は\cite{pmt1}である。
エッセンスだけがかかれているので詳細はすこし分かりにくい。。


PMT法は以下の考えに基づいている。
\begin{itemize}
\item[(1)]
3成分表示での取扱い:\\
基底関数,電子密度、一体ポテンシャル(そしてKSハミルトニアン)を、
「smooth part + onsite part」に分割して
取り扱う手法である。ここで、oniste part =``true part'' - ``counter part''
であり、Muffin-tin(MT)内でのみ値をもつ。
couter part は、smooth partのMT内成分を打ち消す成分である.
smooth partは全空間に広がっており,
onsite成分は原子のサイト(MTサイト)内でのみ値をもつ成分である。
大雑把に言えば、「smooth part + true part - counter part」
の3成分によって表現すると言ってもよい。これを3成分表示の理論形式として
与えることができる。形式論としては、MT半径を重なるようにとっても数学的な破綻は生じない.
また、物理量を計算するときに、これらの3成分の間のクロスタームを
用いないで計算することになる。原子に働く力などもきれいなformalismで計算できる。
このアイデアは現代の線形化法におけるスタンダードのひとつであり、もとは
\cite{soler89}などで与えれている(このときはLAPW法をベー
スに議論している)。PAW法(\cite{PAW},\cite{kresse99})も同じアイデアにもとづいている。

基底関数がなんらかの形で与えれている(たとえば以下の(2),(3)のよう
に)とすれば、波動関数は、それの線形結合で書かれることになる。全エネルギー
は、(占有された)波動関数が与えられれば決定される。解くべき問題は、
その線形結合の係数をうまく選んでその全エネルギーを波動関数の直交性のもとで最小化することである。

%この分割には不定性がある(たとえば、電子密度に関して
%smooth partとcounterパートに同時に同じだけつけくわえても、
%物理的には同じ電子密度をあらわすはずである).
%これを利用して静電エネルギーなどはうまく定義することができる。

%この方法では一体ハミルトニアンにおいて、低い$l$チャンネル
%に関してのみaugmentすることになる。高い$l$成分はaugmentされずにMT内に存
%在していることになる。ただし、

\item[(2)]
基底関数を構成するため、envelope関数として何をもちいるか?:\\
基底関数を用意するには、まずはenvelope関数を用意しそれを何らかの方法で
augmentする。PMT法においては、smooth Hankel関数(従来のFP-LMTOにお
けるenvelope関数)とPlane wave(LAPWで用いられる平面波)の二種をもちいる。
これらをaugmentすることでMTOとAPWが得られる。
固体中において、「平面波的な波動関数(s,p電子など)」と「局在性の高い電
子(d,f電子など)」が共存することを考えると自然な方法であるといえる。
実際、効率の高い方法となる。

\item[(3)]
Augmentation:どうやってenvelop関数から基底関数をつくるか?:\\
Augmentationでは、a.どうやってenvelope関数からcounter partを作るか?
b.切り取った代わりになにを付加するか?の二点が問題となる。PMT法とPAW法の
おもな違いはこの点にあるといえる。

a.PMT法では、smooth Hankel関数の中心のMTにある部分(head part)に関しては,
smooth Hankel関数をそのままcounter partとして用いる。
また、その中心でないMTにある部分(tail part)については、
「ラゲール関数$\times$Gaussianの関数系」で展開し、counter partを作っている。
APW平面波については、このtail partと同様の関数系でそのMT内成分を展開している。

b.基本的には、従来のLAPWにおけるtraditionalなaugmentationの仕方を
もちいる。すなわち、envelope関数に対して、$\phi,\phidot$(各$l$ごとに適当なエネ
ルギー(通常は占有バンド重心位置)で解いた原子基底)を
もちいてMT端での値と微分値が一致するようにaugmentする。


また、セミコア基底関数を、local orbital(MT内に局所化する)として取り込むことも
できるようになっている。これはしばしば必要となる。

\end{itemize}

PAWの文献\cite{kresse99}は、ノーテーションが煩雑だがまとまりはいい。
その記述の多くの部分は、(1)に関係しており、その点ではlmfと共通である.
また、一般論として読むなら、lmfの文献\cite{lmfchap}は、
ちょっとわかりにくい点もある(とくに、Forceをどうやって求めて
るのか?でMethfessel and van Schilfgaardeの文献(\cite{molforce})
が引用されてるが,かなりわかりにくい。
以下では、おもにはlmfの文献を参照しながらlmfのFormalimを説明する
ここでは、全エネルギー最小化で方程式を導出する点で、
上記のlmfの主文献とはすこし違っている。

PAW法とPMT法ではエネルギー汎関数の形や表現法はほとんど同じであり、
用いる基底関数の作成法が違っているだけである。


\section{旧:基底関数に関する古いノート} % ---
\label{sec:oldbasis}
この章は、以前に書いたもので原則的には読む必要がない。
以下の3-component formalismの章にまとめられている。
ただ、記録と、日本語でかかれていることの優位性、もあるので残しておく。
ざっと目を通しおいてもよい。すこしだけコードに関する記述もある。
ーーーーーーーーーーーーー

以下では「ecal/lmfK7]にimplementされているPMT法について述べていく。
ここでは系統性が明瞭になるように独自のノーテーションをもちいるので他文献
との比較には注意を要する。

%%%%%%%%%%%%%%%%%%%%%%%%%%%%%%%%%%%%%%%%%%%%%%%%%%%%%%%%%%%
\subsection{基底関数(augmented wave)の表式} 
最初に全空間をMuffin-Tin(MT)領域と外部領域に分割する。
エンベロープ関数を用意して、それをaugmentすることで、「基底関数」をくみ
たてる。おおまかにいえば、MT内部でのみエンベロープ関数をさしかえるということである。
envelope関数として、PMT方法では「(LAPWでもちいられる)平面波」と、
「(LMTOでもちいられる)smooth Hankel関数」を同時に用いる.
envelope関数はまとめて$\{{F}_{0i}(\bfr)\}$で表す。ここで$i$は、envelope関数を指定するindexである。
「基底関数(augmented wave)」 $\{{F}_i(\bfr)\}$は、
envelope関数$\{F_{0i}(\bfr)\}$を用いて以下のように書かれる。  
\begin{equation}
F_i(\bfr) = F_{0i}(\bfr) + F_{1i}(\bfr)-F_{2i}(\bfr) \label{eq:basis} \\
\end{equation}
ただし、
\begin{eqnarray}
&&F_{1i}(\bfr)=\sum_a F_{1ia}(\bfr-\bfR_a)= \sum_{akL} C^{i}_{akL} \tilde{P}_{akL}(\bfr-\bfR_a) \\ 
&&F_{2i}(\bfr)=\sum_a F_{2ia}(\bfr-\bfR_a)= \sum_{akL} C^{i}_{akL} {P}_{akL}(\bfr-\bfR_a)  
\end{eqnarray}
である(これは、文献\cite{lmfchap}の式19である)。$a$はatomic sites
index. $L=(l,m)$.$F_{1ia}(\bfr),F_{2ia}(\bfr),P_{akL} (\bfr),\tilde{P}_{akL} (\bfr)$ は 
$|\bfr| \le R_a$で定義されている。ここで$R_a$はMTサイト$a$の半径。
 
%これはMTをつかう方法における標準的な波動関数の表式である;
この\req{eq:basis}は、PAWでもLAPWでもLMTOでも同じであり、MTをつかう方法における波
動関数の一般的な表式である.このノーテーションでは、\req{eq:basis}のよう
に基底関数$F_i$は、$F_{0i}$,$F_{1i}$,$F_{2i}$の成分からなっており、それぞれを
第0,1,2成分と呼ぶ。で、「index $i$で指定される基底関数の第0成分をenvelope関数と呼ぶ」と取り決
めたことになる.
%$i$番目のenvelope関数は$F_{0i}(\bfr)$であらわす。この

$F_{0i}(\bfr)$のMTサイトでの成分が$F_{2i}(\bfr)$である
(「各MTサイトにおいて適当な関数系$\{{P}_{akL}(\bfr)\}$を用いて
$F_{0i}(\bfr)$を展開したもの」をすべてのサイトについて加え合わせたもの)。

それゆえ、$F_{0i}(\bfr)+F_{1i}(\bfr)-F_{2i}(\bfr)$は、
$F_{0i}(\bfr)$からMTサイトでの寄与をくり抜いたうえで
$F_{1i}(\bfr)$をつけくわえたものとなっている。
ここで、$F_{1i}(\bfr)$において用いられている
${P}_{akL}(\bfr)$は、各サイトでのradial schoredinger eq.の解
$\phi,\phidot$の線形結合であり、MT端において、値と微分値が
$\tilde{P}_{akL}(\bfr)$と一致している.すなわち、
\req{eq:basis}は、$F_{0i}(\bfr)$
を各MTサイト内においてのみ修正をほどこしたものであり、
「Envelope関数$F_{0i}(\bfr)$をaugmentして基底関数$F_i(\bfr)$が得られた」ことになる。

%それの第0成分である$F_{0i}(\bfr)$がもとになるenvelope関数である。
$F_{1i}(\bfr)$,$F_{2i}(\bfr)$は$F_i(\bfr)$の第1、第2成分であり、
augmentationに関連した成分である;これらは各原子サイト$a$からの寄与の
総和であり、$F_{1i}(\bfr)=\sum_a F_{1ia}(\bfr-\bfR_a)$などと書いたりもし
てする。うえの式と見比べ
ると、$F_{1ia}(\bfr)=\sum_{kL} C^{i}_{akL} \tilde{P}_{akL}(\bfr)$
である(indexの付け方に注意すること)。
%$F_{2i}(\bfr)$は$F_{0i}(\bfr)$のMTへのプロジェクションであり($F_i(\bfr)$を
%MT領域で切り取ったもの)、線形なprojection演算子${\cal P}_2$で、$F_{2i}={\cal
%P}_2F_{0i}$などと書くことも可能である。
このプロジェクションは完全な切り取りにはなってなくて、$L$と$k$の範囲が限
定的である($l\le$LMXA, $k\le$KMXA.$k$は動径方向の自由度のindex)。
すなわち、$F_{0i}(\bfr)$を、各MTサイトで有限個の${P}_{akL} (\bfr)$で展
開し切り取っている。それらを$\tilde{P}_{akL}(\bfr)$で差し替えている。
どうしても展開の不完全さはのこるので、$F_{2i}$は$F_{0i}$を完全なMT内での
成分というわけにはいかない点はある。数値計算の上では、そのことが、計算の
不安定性を引き起こさないようにしないといけない。

\begin{quote}
[注:ノーテーションに整合性を持たせるのが大変である。\cite{PAW}や
\cite{kresse99}ではチルダやらハットやらがごちゃごちゃとした意味で使われ
ている。ここではできるだけすっきりしたノーテーションをもちいるように試
みるがそのため従来の記法とすこし違うところがある。
%以下では0,1,2をそれぞれ「smooth part」, 「onsite true part」,
%「oniste counter part」の表示に用いる。これはlmfのreferenceとはちょっと
%ちがっている。
チルダは後述のように「多重極をGaussianで補正した電子密度$\tilde{n}$
に対して用いるが、上述の$\tilde{P}_{akL}$についても歴史的理由で用いることにする。]
\end{quote}
% However, $F_i(\bfr) = \sum_{akL} C^{(i)}_{akL} {P}_{akL}(\bfr-\bfR_a)$
% for $|\bfr-\bfR_a| <R_a$ is not perfect 
% becasue of cutoff $k<k_{\rm max}$ and $l<l_{\rm max}$.
%$\tilde{P}_{akL}(\bfr)$ is the augmentation atomic functions which replace
%${P}_{akL}(\bfr)$ in the site $a$.


\subsubsection{\bf ${P}_{akL}(\bfr)$の決め方:}
APWの場合\cite{soler89}では、envelope関数$F_{0i}(\bfr)$は平面波であり、
$F_{0i}(\bfr)$のMTサイトでの展開の$L$成分は、エネルギー$|\bfq+\bfG|^2$の球
ベッセル関数$\times Y_L$となる。lmfではこの球ベッセル関数を
ラゲール多項式を修正した関数系$p_{kl}(\gamma,r)$で展開している(ラゲール多項式の
引数に$r^2$を代入したものであり,\cite{Bott98}の12.15式あたりに詳細な説明
がある)。この$k$の範囲が$0\le k \le$KMXAとなる。

MTOではすこし複雑なことをしていてsmooth Hankelを
augmentするとき中心の原子におけるMT(Head Part MT)では
各$L$ごとに完全にradial partをくりぬくやり方でaugmentしている。
すなわち、このときには${P}_{akL}(\bfr)$として、smooth Hankelのonsite成
分そのものを使っていることになる。そのためkの数は一個ですむ。
またTail partのMTでは、上述の$p_{kl}(\gamma,r)$で展開している。
\cite{lmfchap}のp.11。\\


[$p_{kl}(\gamma,r)$での展開は再検討してもいいかもしれない。
これはbndfp.Fのhambl-augmbl.F-bstrux-pauggp
あたりに関係している。bndfpでよばれるhamblはハミルトニアンをつくるサブルー
チンであり、napwがAPWの数であり、kmaxがKMXA。とくに、(推定だが)問題になりそうなの
は、もとの平面波のMT内での絶対値の2乗和より、projectしたものの2乗和の
ほうが大きくなりうる点である。これはPAWでも起こりうるが、ノルムがマイナ
スになるような基底をあたえてしまう要因になりうる。]


\subsubsection{\bf $\tilde{P}_{akL} (\bfr)$の決め方:}
まず、各$a,l$に対してradial schoredinger eq.を
特定のエネルギーenu(もしくはそれに対応した対数微分であるP(pnu)で解く。
そのあと、それのエネルギー微分をつくる。これらがphi,phidot
($\phi_{al}(r),\dot{\phi}_{al}(r)$)である。
次に、それらの線形結合
$\tilde{P}_{akL}(\bfr)=(A\phi_{al}(r)+ B \dot{\phi}_{al}(r))\times Y_L$
を考える。この際、係数$A,B$は、MT端において
${P}_{akL}(\bfr)$と同じ値と微分値をもつように決定する。

Pの値はコンソール出力でptryで表示されている。
(一方、コンソール出力で表示されるebarは占有軌道の各lごとのDOSの重心位置。
現在のデフォルトはOPTIONS PFLOAT=1である。
古いバージョン(PFLOAT=0)ではバグがあった;過去とのcompatibilityのための
こしてある)。通常,ebarに対応するptryを計算しそれに対応したptryをpnewとする、
(コードではfp/pnunew.F L92;ebar =hbyl(m,isp,ib)/qbyl(m,isp,ib)参照)。
しかしそれでは不十分で、自由空間での対数微分の値(fp/pnunew.F;Free electron value)より深くなりす
ぎないという制約を課している;これは単純にはMT内でのポテンシャルが
上に凸であることを意味する。占有数が少なすぎるとき、
hybridyzationの影きょうで見かけ上そういうこと
(高い位置にあるべきenuが下がりすぎることが起こる)。

%In the one-body problem for given one-body potential $V(\bfr)$,
%we consider the eigenvalue problem with the Hamiltonian and overlap matrix 
%$\langle F_i| -\frac{\nabla^2}{2m} |F_j \rangle $ and 
%$\langle F_i| F_j \rangle $. However, this procedure contains complicated
%cross terms. Instead of this procedure, we consider another type of one-body problem
%explained in the following.

$\tilde{P}_{akL}(\bfr)$は、phi,phidotという関数で表現されており、
self-cosnsitencyへ達するitetationにおいて更新されていくのがデフォルトだ
が、固定することも出来る(FRZWFオプション,IDMOD=1、将来的には原子で計算し
たphi,phidotに固定したほうが系統的誤差を減らせる可能性がある)。


%%%%%%%%%%%%%%%%%%%%%%%%%%%%%%%%%%%%%%%%%%%%%%%%%%%%%%%%%%%
\subsection{基底関数の張る空間と関数の積} 
\req{eq:basis}のように$F_i$は3つの成分からなっていた。
%$F_i=[F_{0i}(\bfr),F_{1ia}(\bfr), F_{2ia}(\bfr)]$。ここで、
%$F_{1ia}(\bfr) =\sum_{kL} C^{i}_{akL} \tilde{P}_{akL}(\bfr)$であり、
%$F_{2ia}(\bfr) =\sum_{kL} C^{i}_{akL} {P}_{akL}(\bfr)$. 
%3つの成分を第0成分、第1成分、第2成分と呼ぶことにする。
%$F_{1ia}(\bfr)$と$F_{2ia}(\bfr)$については$|\bfr| \le R_a$.\\
%\noindent (ユニットセルに一個以上のMTサイトがある時には
%$F_i= [F_{0i}(\bfr), \{ F_{1ia}(\bfr), F_{2ia}(\bfr); 
%a\in {\rm atomic \ sites} \}]$と書いた方が正確だが、単純化して上のよ
%うに書くことにする
%, but we sometines use the above expression forsimplicity
%).
単純には\req{eq:basis}のように$F_i=F_{0i}+F_{1i}-F_{2i}$と書けるが、この意味には注意を
払う必要がある.もともとは単純な足し算の意味であったが、
実際的には、これらの個々の成分は独立性をもたせて扱うので、(もと
もとの意味を忘れてはいけないが),数学的にwell-definedなものとするには、
$F_i$は独立な3成分からなるものであると考える必要がある。実際、
$F_{0i},F_{1i},F_{2i}$を直接に加減するような演算はでてこない。

%ちょっとおかしい。
%;たとえば、ある関数を$F_{0i}$に加え、その
%分を$F_{2i}$から差し引くという変換をしても物理的には等価なものを表すこと
%になる;全エネルギーはそういう変換に対する不変性をもつ必要がある(注:議
%論をもう少し加えること)。

%
%言い方を変えると以下のようにも言える。
%「ヒルベルト空間(複素関数空間)
%${\cal W}_{\rm all}={\cal V} \otimes \Pi_a ({\cal A}_a \otimes {\cal A}_a)$
%(${\cal V}$は全空間であり、${\cal A}_a$はMTサイト$a$の複素関数空間)を考えると、
%基底関数(basis function)の張る空間 ${\cal W}=\{ F_i \}$は、
%${\cal W}_{\rm all}$の部分空間となる。」関数の連続性が係数$C^{i}_{akL}$
%で規定されている。
%
%
%${\cal W}$は、可能な限り最適であり、かつMinimumであり、
%かつtranferabilityも十分に備えている必要がある。我々は、
%「未来の電子状態計算においては局在的なenvelope関数と,平面波的なenvelop関
%数を同時に必要とするであろう」と考えPMT法を開発することにした。


%おり、その部分空間の元を「空間連続性をもった関数である」と呼ぶことができる。

のちほど、Hamiltonianや重なり積分を考える時には
波動関数の積を考える必要があるが、
$F_i$は3つの成分からなっているので、
式\req{eq:basis}を単純に2乗するとクロスタームを考えて3x3=9
つの項がでてくる。しかし、実際にはクロスタームは無視して取り扱う。
(ただし$k_{\rm max},l_{\rm max}$が無限大の極限では厳密に無視できる)。
すなわち、基底関数で張られる空間${\cal W}=\{ F_i \}$においては二つの関数の積を3成分の表示で、
\begin{eqnarray}
F_i(\bfr) F_j (\bfr)
&=& F_{0i}(\bfr) F_{0j}(\bfr)+ F_{1i}(\bfr)F_{1j}(\bfr)-
F_{2i}(\bfr)F_{2j}(\bfr)
\label{eq:product}
\end{eqnarray}と定義する(これは電子密度や波動関数の積において用いる。
和は形式的なものであり、3つの成分は独立したものであると考える必要がある)。
また、空間積分は
\begin{eqnarray}
\int F_{0i}(\bfr) F_{0j}(\bfr) d^3r 
+\sum_a \int_a F_{1ia}(\bfr) F_{1ja}(\bfr) d^3r
-\sum_a \int_a F_{2ia}(\bfr) F_{2ja}(\bfr) d^3r
\end{eqnarray}
で定義する。内積とするには$F_i$の複素共役をとる。和は自然に定義できるし、
積や積分が定義できている。それで積としての電子密度(密度行列も)自然につくれる。

クーロン相互作用の定義に関しては、密度も3成分表示されるので、$3\times3$
成分が出てくる。これをそのまま計算することも可能ではあるが、
後述の方法ではいったん多重極の考えを利用して,3成分の電荷密度$n$に対して、
$n\rightarrow \tilde{n}$という変換を定義し、$\tilde{n}$をもちいて
クーロン相互作用を定義することとする。これは波動関数のレベルでも行えるので、
$\langle \psi \psi |v| \psi \psi \rangle$も原理的にはきちんと定義できている
(将来的にこれをきちんと尊重するようにfpgwをなおしていかないといけない)。

$L$のカットオフはLMXA。

{\small [\noindent 注:この括弧内は読む必要なし。修正必要。PAWのformalismでは、
${F}_i={\cal P}F_{0i}$というような形で、
「あるprojector${\cal P}$をsmooth part$F_{0i}$に作用させることにより
Fullの波動関数が得られる」とする。それで、PAWのセールスポイントは
『total energyをfullな波動関数の代わりに「$F_i$の線型結合からなる
smoothな波動関数に関して最小化する」という形で定式化でき、projectorを
たくさんとればこれはいくらでも厳密化できる』というものであった。
しかし、これは正しいとは思えない。projectorはLAPWで行うように2つにとどめるのがよいように思える。
単純には、projectorを増やしていくと、真の解にいきつくより先に、解がこわ
れるということがおこるとおもわれる;昔からLAPW業界ではphi,phidotだけでな
くもうひとついれて解の精度をあげることが試みられてきたが単純にはうま
くいかない。基底をふやすにはlocal orbital、あるいは(違うenuでといたもの
でaugmentする)MTOをいれるべきである;これらの手法では、外部の基底と内部
の基底の接続をすこしちがうやりかたで行うことになる。]}\\


以下では、この3成分をもつ
%部分空間${\cal W}$のなかにある
クーロン相互作用、電子密度と外場との結合を、もともとの波動関数がどうであっ
たかに注意しながら、線形演算子として定義していく。
波動関数に対して、内積、微分($\nabla$)、
さらにLDA計算における全エネルギーを定義し、それの最小化で解をもとめる方
程式をだす。フルハミルトニアンも作れるので,GW計算の式などもおのずと与えられる
($\langle \psi \psi |v| \psi \psi \rangle$が与えられるので)。


%[{\bf ???}ヒルベルト空間${\cal W}$での量子力学がきちんと定義できていると言って
%よいのか???電磁場との相互作用とゲージ不変性から電子密度やカレントを定義
%する。Non-colinearをきちんと導出。]



%また、電子密度、密度汎関数も定義できる。このことは将来的により洗練された形のGW近似などをimplementしよ
%うとするときに重要になってくるかもしれない
%(実際、現在のGWのimplimentationでは、このクーロン相互作用の定義を
%ちゃんと守ったimplementationになってない;微妙に乱暴になっている。要は
%$\langle \psi_i\psi_j|v|\psi_k \psi_l\rangle$をどう計算するか?ということ)。

% (電荷保存法則とかには注意しないといけないかも.電磁場までいれた量子力学を
% 書き下しておいたほうがいいかも。誰か。。).

% We treat the quantum mechanichs in the Hilbert space $\{ \tF_i \}$
% instad of the full problem in real space. It is
% treated numerically very accurately. In order to define the quantum mechanichs, 
% we have to fix the definitions product of eigenfunctions, $\nabla$, norm, 
% and the Coulomnb interaction. 
% We will show how the one-body problem in LDA in represented in ${\cal W}$.
% The quantities in real space can be simply identified in the $W$ space.

%とにかく、このような「きちんと組み立てておく」というのが大事なような気が
%する。そうしておくと数値計算がしっかりした土俵に乗る。こういう
%できるだけおおきい枠組みのレベルでモデル化してカットオフを導入し
%ておくことをしておかないといけない。
%低レベルでへんなカットオフをいれるようなコードだと、制御不能になってくる。


%またそういうことを意識しておくとコードを書くときにもどこに注意を払うべきか?とかがわかる。
%最小化の原理がどう成り立つかとか、たとえばある項を意図的に固定するとどう
%振る舞うか?などが予想できないとデバッグも困難である。


%また、系統的なエラーを除くには、コードが最小化原理にそって組み立てられて
%ることが重要であるとおもわれる。


%\noindent {\bf 注意}:もともとのenvelope関数$F_i(\bfr)$から、
%$\sum_{akL} C^{i}_{akL} {P}_{akL} (\bfr)$をとりだす操作はプロジェクショ
%ンといわれる。LAPW、PAW、LMTOなどではインプリメンテーションによりいくつかの
%種類のプロジェクションの仕方がありえる。


%以下の議論において「3成分基底関数$F_{i0}$,$F_{i1}$,$F_{i2}$の線形結合で物理
%量をあらわす」という意味においては、数学的にきちんと定義をしていってるの
%で、3つの成分を独立なものであると思っても、内積が正定値であるかぎりにおいては、
%問題なくwell-definedになっている。その意味でMTがかさなっても数学的には破
%綻していない.


以下の議論をきちんとみていくと、
「3成分ヒルベルト空間の有限次元の部分空間であって、内積
$= \int dr F^*_{0i}(\bfr) F_{0j}(\bfr)  
+ \sum_a \int_{|\bfr| <R_a} d\bfr (F^{*}_{1i}(\bfr) F_{1j}(\bfr) -F^{*}_{2i}(\bfr) F_{2j}(\bfr))$
が正定値になっているもの」において量子力学が数学的にきちんと定義されてい
ることがわかる。基底はとにかくまず固定するわけでありtigh-bindingモデルで
の量子力学である。

MTが重なっていても数学的な破綻は生じない。
ただ、これが「もともとの問題を正確にモデル化したもの」になってるかどうか
は別問題である。この点で$\{ F_i \}$のクオリティが問われることになる。

「3成分で表現された基底をどうつくるか?」という問題と
「もし基底があたえられたとしてそれからどうLDAの全エネルギー関数(あ
るいは量子力学)をset upするか?」は別問題である。あるいは、ここを分離し
て考えるところがキーポイントである。

基底関数が固定されていれば、最終的に得られるLDAでの全エネルギーは、tight
bindingなもの(基底関数の線型結合の係数を選んでから波動関数を決定する問
題)となる。それで、$F_{i}$の3成分が独立に与えられているとしても、数学的
にきちんと定義された問題として成り立っている
%これは数値計算のrobustnessの点から非常に重要である。
%この部分でしっかりしたコーディングになってないといけない。
%
%実際には、iterationの途中で$\tilde{P}_{akL}$の最適化
%もおこなっていくことになる(これは全エネルギー最小化から直接に出て
%くる最適化ではなく、PDOSの重心位置でradial schoredinge eq.を解いて
%$\tilde{P}_{akL}$をきめる)。


%;$\{F_{i}\}$の内積が正定値であれば、$\{F_{2i}\}$が
%$\{F_{0i}\}$のプロジェクションである必要もない。


\subsection{重なり積分}
前章で定義した内積から、
%We introduce an inner product of ${\cal W}$, where 
%these three type of components are orthogonal each other.It is (overlap matrix is) 
\begin{eqnarray}
&&O_{ij} \equiv \langle F_i| F_j \rangle 
= \int dr F^*_{0i}(\bfr) F_{0j}(\bfr)  
+ \sum_a \int_{|\bfr| <R_a} d\bfr (F^{*}_{1i}(\bfr) F_{1j}(\bfr) - F^{*}_{2i}(\bfr) F_{2j}(\bfr)) \nonumber \\
&&= \int dr F^*_{0i}(\bfr) F_{0j}(\bfr)  \nonumber \\
&&+ \sum_a \sum_{L} (C^{i}_{akL})^* \sum_{k'} C^{j}_{ak'L}
\left(\int_{|\bfr| <R_a} d\bfr \tilde{P}_{akL}(\bfr) \tilde{P}_{ak'L}(\bfr)
-\int_{|\bfr| <R_a} d\bfr {P}_{akL}(\bfr) {P}_{ak'L}(\bfr) \right)
\label{eq:over}
\end{eqnarray}
となる。基底関数$F_{i}$は、その成分
$F_{0i},C^i_{akL},\tilde{P}_{akL}(\bfr),{P}_{akL}(\bfr)$
によって指定されている。
%また、この内積には原子の位置$\bfR_a$は直接には含まれていない。


\subsection{重なり積分が正定値になっているか?}
前述の重なり積分は、$F^{*}_{2i}(\bfr) F_{2j}(\bfr)$に関わる部分で負符号
になりうる部分を含んでいる。原則的には、
\begin{eqnarray}
\int dr F^*_{0i}(\bfr) F_{0j}(\bfr)  
+ \sum_a \int_{|\bfr| <R_a} d\bfr ( - F^{*}_{2i}(\bfr) F_{2j}(\bfr)) \nonumber 
\label{eq:over2}
\end{eqnarray}
が、正定値性を満たしているのが望ましいと思われる。しかし、
PAWや現在のlmfではそれは保証されていない。検討の余地がある。

{\small [ひとつの考え方としては、各$a$サイトにおいて、
$\int_{|\bfr| <R_a} d\bfr F^*_{0i}(\bfr) F_{0i}(\bfr)  
>\int_{|\bfr| <R_a} d\bfr F^{*}_{2i}(\bfr) F_{2i}(\bfr)$ 
となるようにとると、これを自然にみたすことができる。
すなわち各サイトにおいて、もとのenvelope関数の二乗積分のほうが、それを射影して得た$F_{2i}$の二
乗積分よりおおきくなっていればよい。
このためには、$F_{2i}$を切り出すprojectorの関数が正規直行系であればよい。
検討の余地がある。]}

%%%%%% --------------- PMT formalism ------------------
\section{3-components formalism}
\label{sec:formalism}
We assume periodic boundary condition where
real space (or unit cell) is specified by $\Omega$.
$\Omega$ is divided into the muffin-tin (MT) regions and the interstitial region.
The MTs are located at $\bfR_a$ with radius $R_a$, where 
$a$ is the index specifying a MT within $\Omega$.
$L\equiv(l,m)$ is the angular momentum index.
We use units, $\hbar=1$, electron mass $m_e$, and electron charge $e$. 
The spin index is not shown for simplicity.

Here we give the 3-component formalism as a general
scheme for the augmented-wave methods, which include any kinds of
augmented waves including the local orbitals \cite{PhysRevB.43.6388}.
%The 3-component formalism is given from the point of view that 
%how to model the quantum mechanics in the 3-component space as follows.

(訳)一般的なLAPW法などと同様に、実空間における周期境界条件を考える。
実空間(もしくは実空間におけるユニットセル)を$\Omega$であらわす。
$\Omega$はMTregionとその外側のinterstitial regionに分かれる。
MTは原則的には重なっているべきではない(しかしながら、以下のformalismでは
MTが重なっていても数学的破綻はなく、DFT計算においてはいくらかの重なりがあっ
ても、それほどおかしくない全エネルギーの値が得ることはできる)。
MT半径は$R_a$でその中心は、$\bfR_a$にあるとする。ここで$a$は$\Omega$内のMTを指定するindexである。
角運動量のindexにはcombined angular momentum index
$L\equiv(l,m)$をもちいる。ここで $l$ と $m$ はそれぞれ、orbital と magnetic quantum
numbersである。また$\hbar=1$, electron mass $m_e$, and electron charge
$e$を用いる。以下では、スピン変数については簡単のため省略して、formalism
を書き下す(適宜おぎなって読んでいく必要がある)。

%%%%%%%%%%%%%%%%%%%%%%%%%%%%%%%%%%%%%%%%%%%%%%%%%%%%%%%%%%%%% 
\subsection{the 3-component space}
\label{sec:3compo}

(日本語による蛇足的説明)
Augmentされた基底関数$F_{i}(\bfr)$は3つの成分からなっている。それらは、
the smooth part (= envelope function) $F_{0i}(\bfr)$;
the true parts in MTs $F_{1i,a}(\bfr)$
; the counter parts in MTs (canceling the smooth part) 
$F_{2i,a}(\bfr)$である。PMT法では、$F_{0i}(\bfr)$は、
\smh\ $h_L(\bfr-\bfR_a)$のBloch sumあるいはPWである。
$F_{1i,a}(\bfr)$と$F_{2i,a}(\bfr)$はMT内部$|\bfr-\bfR_a|<R_a$でのみ定
義されている(以下の数式においては、場合により$|\bfr-\bfR_a|>R_a$でゼロであると考える)。
従来のaugmentationの理論においては、これらを用いて基底関数は
$F_i(\bfr) = F_{0i}(\bfr)+\sum_a F_{1i,a}(\bfr-\bfR_a)- \sum_a F_{2i,a}(\bfr-\bfR_a)$ 
(e.g. See Eq.(2) in \cite{kresse99})と書くことができる。
しかしながら,以下のthe 3-component augmentation for bilinear productsの
方法では、この表式は形式的なものであり、各コンポーネントは
それぞれ個別に処理され、コンポーネントをつなぐような項は現れ得ない。
これを考えると、基底関数$F_i$はむしろ「関数の集合」であると考えたほうが
自然であり、$F_i=\{F_{0i}(\bfr),\{F_{1i,a}(\bfr)\},\{F_{2i,a}(\bfr)\} \}$
という表式を用いて理論をもちいて書き下したほうが、
より正確に理論が表現でき、以下で見るように数学的に素直で論理構造が明瞭なformalismが組み立てられる。
ただ、上の$F_i$の表式をそのまま使っていくと、はやたらと括弧がやたら多く、
見にくい数式になってしまう。それで以下のように組み立てていく。\\


Any augmented basis $F_{i}(\bfr)$ consists of three kinds of components, 
where $i$ is the index specifying basis function. $F_{i}(\bfr)$
consists of the following three components:
\begin{itemize}
\item[(0)] 
the smooth part (= envelope function) $F_{0i}(\bfr)$
\item[(1)]
the true parts $F_{1i,a}(\bfr)$ defined in MTs $|\bfr| \leq R_a$.
\item[(2)]
the counter parts $F_{2i,a}(\bfr)$ defined in MTs $|\bfr| \leq R_a$ 
(canceling the smooth parts within MTs).
\end{itemize}
We call $F_{0i}(\bfr)$, $F_{01,a}(\bfr)$, and $F_{02,a}(\bfr)$ as the
0th, 1st, and 2nd components of $F_{i}(\bfr)$, respectively.
The $F_{0i}(\bfr)$ should be an analytic function in space or a linear
combination of analytic functions.
In the PMT method, $F_{0i}(\bfr)$ are the PWs or the Bloch sums of the
Hankel functions; exactly speaking, we use atom-centered 
smooth Hankel functions (\smhs) instead of the conventional Hankel functions, 
so as to avoid divergence at its center 
\cite{lmfchap,Bott98} (See \req{eq:defh0} and around). 
$F_{1i,a}(\bfr)$ and $F_{2i,a}(\bfr)$ are defined only at
$|\bfr| \leq R_a$ (in cases below, we sometimes take these are zero at $|\bfr|>R_a$).
In the sense that $F_{0i}(\bfr)$ is analytic and 1st and 2nd components 
are given on a dense radial mesh, a basis $F_{i}(\bfr)$ is specified
without any numerical inaccuracy.

$F_{i}(\bfr)$ is a member in the {\it 3-component space}, which
is defined as a direct sum of linear spaces
corresponding to the components (0), (1) and (2). Thus $F_i$ can be expressed as
$F_i=\{F_{0i}(\bfr),\{F_{1i,a}(\bfr)\},\{F_{2i,a}(\bfr)\} \}$
(curly bracket mean a set), where $F_{1i}\equiv \{F_{1i,a}(\bfr)\}$
means a set as for the MT index $a$, $F_{2i}$ as well. However, in the followings, we use a
little different expression instead:
\begin{eqnarray}
F_i(\bfr) = F_{0i}(\bfr) \ooplus \{ F_{1i,a}(\bfr)\} \oominus \{F_{2i,a}(\bfr)\}\nonumber\\
= F_{0i}(\bfr) \ooplus F_{1i}(\bfr)\oominus F_{2i}(\bfr).
\label{eq:basis}
\end{eqnarray}
This makes following expressions easy to read without any difference
in their meanings. Symbols $\ooplus$ and $\oominus$ mean nothing more
than separators.
We call a member in the 3-component space as a 3-component function in
the followings.
Wavefunctions are also given as 3-component functions.
With the coefficients $\{\alpha_{p}^i\}$, wavefunctions can be written as
\begin{eqnarray}
\psi_p(\bfr) = \sum_i \alpha_p^i F_i(\bfr),
\label{eq:eig}
\end{eqnarray}
where linear combinations are taken for each components.
We represent electron density and so on as a 3-component function as well.

Note that the 3-component space is a mathematical
construction, a model space: we have to specify how to map a 3-component 
function to a function in real space.
For this purpose, we define $\calR$-mapping (augmentation mapping)
from a 3-component function to a function in real space;
\begin{eqnarray}
\calR[\psi_p(\bfr)]\!\equiv\! \psi_{0p}(\bfr)
\!+\! \sum_a \psi_{1p,a}(\bfr\!-\!\bfR_a)\!-\!\sum_a \psi_{2p,a}(\bfr\!-\!\bfR_a). \nonumber \\
\label{eq:calRF}
\end{eqnarray}
This is nothing but a conventional augmentation 
where physically meaningful wavefunctions $\psi_{p}(\bfr)$ 
should satisfy following conditions (A) and (B);
\begin{itemize}
\item[(A)]
Within MTs ($|\bfr|<R_a$), $\psi_{2p,a}(\bfr)=\psi_{0p}(\bfr+\bfR_a)$.
\item[(B)]
At MT boundaries ($|\bfr|=R_a$), $\psi_{1p,a}(\bfr)$ and $\psi_{2p,a}(\bfr)$
should have the same value and slope.
\end{itemize}
If (A) is satisfied, the contribution from $\psi_{0p}$ within MTs 
perfectly cancels those of $\psi_{2p,a}$ in \req{eq:calRF}. 
The total energy in the DFT is given as a functional
of eigenfunctions as $E[\{\psi_{p}(\bfr)\}]$, where
$\{\psi_{p}(\bfr)\}$ are for occupied states.
Our problem is to minimize this under the constraint of
orthogonality of $\psi_{p}(\bfr)$ with conditions 
(A) and (B) on $\{\psi_{p}(\bfr)\}$.
Local orbitals \cite{PhysRevB.43.6388} is also treated as 3-component
functions whose 0th and 2nd components are zero overall.

In the conventional LAPW (e.g. See \cite{bluegel31,Singhbook}),
(A) and (B) are very accurately satisfied. The 2nd component
almost completely satisfy (A) with the use of spherical Bessel functions.
The 1st component are given up to very high $l$ ($\gtrsim 8$). 
Thus the LAPW can be quite accurate. However, it can be 
expensive (we also have null-vector problem. See \refsec{sec:problems}.).

Thus Soler and Williams \cite{soler89} introduced additive augmentation:
to make calculations efficient, we use condition (A') as a 
relaxed version of condition (A),
\begin{itemize}
\item[(A')]
Within MTs ($|\bfr|\leq R_a$), 
$\psi_{2p,a}(\bfr) \approx \psi_{0p}(\bfr+\bfR_a)$.
\end{itemize}
Then we expect high-energy (high frequency) contributions
of eigenfunctions not included in the 1st and 2nd components 
are accounted for by the 0th component. In practice, we can use low $l$ 
cutoff $\lesssim 4$ for both of 1st and 2nd components. 
A LAPW package HiLAPW, developed by Oguchi et al \cite{PhysRevB.54.1159},
implemented a procedure to evaluate physical quantities from 
the basis given by \req{eq:calRF} with the condition (A').

However, it is complicated to evaluate all quasilocal 
products such as the density and kinetic-energy density
from $\calR[F^*_i(\bfr)]\calR[F_j(\bfr')]$,
since it contains cross terms which connect different components.
Thus Soler and Williams \cite{soler89} 
gave a prescription to avoid the evaluation of the cross terms.
With $\calR$-mapping applied not to wavefunctions
but to products of them as in \refsec{aug3},
we have {separable form} of the 
total energy and all other physical quantities 
(no cross terms between components). This is based on the fact that
the total energy in the {separable form} should 
agree with the true total energy only when (A) and (B) are satisfied. 
As we see in the followings, it is a good approximation to use (A') instead of (A). 

Above two important concepts, 
the additive augmentation and the separable form, 
were used in both of LMTO and PAW \cite{lmfchap,PAW,kresse99}.
They were originally introduced in Ref.\onlinecite{soler89}.

Let us consider how to determine $F_{1i,a},F_{2i,a}$ for
a given $F_{0i}$. As for $F_{2i,a}$, (A') means that 
$F_{0i}$ should be reproduced well within MTs.
Generally speaking, $F_{2i,a}(\bfr)$ can be represented as
\begin{eqnarray}
&&F_{2i,a}(\bfr) \equiv \sum_{k,L} \CiakL P_{akL}(\bfr), \label{f2}
\end{eqnarray}
where $k$ is index for radial degree of freedom. 
We introduce truncation parameters
$k_{{\rm max},a}$ and $l_{{\rm max},a}$; we assume 
sum in \req{f2} is taken for $k\le k_{{\rm max},a}$ and $l \le l_{{\rm max},a}$;
when $k_{{\rm max},a}$ and $l_{{\rm max},a}$ becomes infinite, we assume
condition (A) is satisfied. Even when these truncation
parameters are finite, $F_{2i,a}$ should reproduce
low energy (low frequency) parts of $F_{0i}$ well. 
The functions $\{P_{akL}(\bfr)\}$ can be rather general; 
as explained in \refsec{sec:pmtmethod}
the central parts of \smh\ is treated as it is
(in other words, treated as a member of $\{P_{akL}(\bfr)\}$ \cite{privatemark1}).
$F_{1i,a}(\bfr)$ is given from \refeq{f2}
with a replacement of ${P}_{akL}(\bfr)$ with $\widetilde{P}_{akL}(\bfr)$.
Here $\widetilde{P}_{akL}(\bfr)$ is a linear combination of
partial waves so as to have the same value and slope with
${P}_{akL}(\bfr)$ at $|\bfr|=R_a$. With this replacement, we have 
\begin{eqnarray}
F_{1i,a}(\bfr) = \sum_{k,L} C^i_{akL} \wPakL(\bfr). \label{f1}
\end{eqnarray}
%We can usually take relatively small $l_{{\rm max},a}$ ($\sim 4$) in practice.
%as discussed in Ref.\cite{soler89}.

%%%%%%%%%%%%%%%%%%%%%%%%%%%%%%%%%%%%%%%%%%%%%%%%%%%%%%%%%%%%%%%%%%%%%%%%
\subsection{augmentation for product of 3-component functions}
\label{aug3} Let us give a prescription to evaluate physical quantities
for wavefunctions satisfying conditions (A') and (B).  First, we define
diagonal product of 3-component functions as
\begin{widetext}
\begin{eqnarray}
F^*_i(\bfr)F_{j}(\bfr') &\equiv& F^*_{0i}(\bfr)F_{0j}(\bfr')
\ooplus  \{ F^*_{1i,a}(\bfr) F_{1j,a}(\bfr')\}
\oominus \{ F^*_{2i,a}(\bfr) F_{2j,a}(\bfr')\}, \label{fnproduct}
\end{eqnarray}
where we have no cross terms between different components.
We apply $\calR$-mapping in \req{eq:calRF} to this product as
\begin{eqnarray}
&&\calR[F^*_i(\bfr)F_{j}(\bfr')] = \nonumber \\ 
&&F^*_{0i}(\bfr)F_{0j}(\bfr')
+ \sum_a F^*_{1i,a}(\bfr\!-\!\bfR_a) F_{1j,a}(\bfr'\!-\!\bfR_a)
- \sum_a F^*_{2i,a}(\bfr\!-\!\bfR_a)F_{2j,a}(\bfr'\!-\!\bfR_a). \label{fnmap}
\end{eqnarray}
We will use $\calR[F^*_i(\bfr)F_{j}(\bfr')]$ to evaluate quasilocal 
products when (A') is satisfied.
%\req{fnmap} means an augmentation not for the basis itself, 
%but for the product of them.
Since any one-body quantities such as the inner product, electron density,
current and so on, are quasilocal, we can evaluate these from 
$\calR[\psi^*_p(\bfr)\psi_{p'}(\bfr')]$.
Generally speaking, we can evaluate matrix elements of a
quasilocal operator $X(\bfr,\bfr')$ in real space
from 3-component wavefunctions $\psi_p(\bfr)$ in separable form as 
\begin{eqnarray}
\langle \psi_p|X|\psi_{p'} \rangle=\int d^3r d^3r' X(\bfr,\bfr') 
\calR[\psi^*_p(\bfr)\psi_{p'}(\bfr')].
\end{eqnarray}
We can read this as a transformation of $X$ to the corresponding 
operator in the 3-component space.

Based on the above prescription, we can define the inner product 
$\langle \psi_p |\psi_{p'} \rangle$ as $\langle \psi_p |\psi_{p'}
\rangle=\sum_{i,j}\alpha_{p}^{i*}\alpha_{p'}^{j}O_{ij}$.
Here the overlap matrix  $O_{ij}$ is:
%\begin{widetext}
\begin{eqnarray}
O_{ij} &\equiv& \langle F_i|F_{j} \rangle \equiv \int_\Omega d^3r 
  \calR[F^*_i(\bfr)F_j(\bfr)] \nonumber \\
&=&\int_\Omega d^3r  F^*_{0i}(\bfr)F_{0j}(\bfr)
  + \sum_a \inta d^3r  F^*_{1i,a}(\bfr)F_{1j,a}(\bfr) 
  - \sum_a \inta d^3r  F^*_{2i,a}(\bfr)F_{2j,a}(\bfr).  \label{eq:norm} 
\end{eqnarray}
This can read as a definition of the inner product in the 3-component space.
For a given finite basis set, we can expect that $O_{ij}$ should be
positive definite as long as truncation parameters are large enough.
The kinetic energy is given 
from $\rho_{ij}=\sum_{p}^{\rm occ.} \alpha^{i*}_p \alpha^j_p$ (occ. means
the sum for occupied states) as $\ek=\sum_{i,j} \rho_{ij} T_{ij}$.
Here the kinetic-energy matrix $T_{ij}$ is given as
\begin{eqnarray}
&&T_{ij}\equiv \frac{\langle \nabla F_i| \nabla F_{j} \rangle}{2m_e}
 \equiv \frac{1}{2m_e} \int_\Omega d^3r \left(\nabla_\bfr \nabla_{\bfr'}
 \calR[F^*_i(\bfr)F_j(\bfr')]\right)_{\bfr=\bfr'} 
=\frac{1}{2m_e} \int_\Omega d^3r 
\calR[\nabla F^*_i(\bfr) \nabla F_j(\bfr)] \nonumber \\
&&= \int_\Omega d^3r \frac{\nabla F^*_{0i}(\bfr) \nabla F_{0j}(\bfr)}{2m_e} 
+ \sum_a \inta d^3r \frac{\nabla F^*_{1i,a}(\bfr) \nabla F_{1j,a}(\bfr)}{2m_e}
- \sum_a \inta d^3r \frac{\nabla F^*_{2i,a}(\bfr) \nabla F_{2j,a}(\bfr)}{2m_e}.
\label{eq:kin}
\end{eqnarray}
Partial integration gives $T_{ij}= \langle F_i| \frac{-\nabla^2 }{2m_e}|F_{j} \rangle$, 
since $F_{1i,a}$ and $F_{2i,a}$ have the same value and slope at the MT boundaries.
This kinetic energy operator is interpreted as $T=\frac{-\nabla^2 }{2m_e} \oplus
\{\frac{-\nabla^2 }{2m_e} \} \ominus  \{\frac{-\nabla^2 }{2m_e} \}$ in the 3-component space.

One-body problem for a given one-particle potential $V(\bfr)$ in real
space is translated into a problem in the 3-component space for the
Hamiltonian $H=T+V$ under the condition (A) or (A'), where $V=V_0 \oplus
\{V_{1,a}\} \ominus \{V_{2,a}\}$.  Here $V_0(\bfr)=V(\bfr)$, and
$V_{1,a}(\bfr)=V_{2,a}(\bfr)=V(\bfr+\bfR_a)$ within MTs at
$\bfR_a$. However, we can add any extra potential $\Delta \bar{V}$
simultaneously to both of $V_0$ and $V_{2,a}$ if (A) is completely
satisfied.

We have an error because we use \req{fnmap} instead of \req{eq:calRF}:
high energy contributions contained in the 0th components are not
exactly evaluated. However, the error can be small enough to be
neglected as discussed in Appendix \ref{sec:zeroonetwo}.  This error is
also related to a question, how to choose the optimum $\Delta \bar{V}$
so as to minimize the error.  In fact, the success of the PAW \cite{PAW}
is dependent on the choice of $\Delta \bar{V}$ as seen in
\refsec{sec:comparison}.

The valence electron density $n$ as the 3-component function is given by
\begin{eqnarray}
n &=&n_0\ooplus n_1 \oominus n_2 = n_0 \ooplus \{n_{1,a}\} \oominus \{n_{2,a}\} =
\sum_{ij} \rhoij F_i^* F_j = \sum_{ij}
 \rhoij F^*_{0i}(\bfr)F_{0j}(\bfr) \nonumber \\
&&\ooplus  \{ \sum_{ij} \rhoij F^*_{1i,a}(\bfr)F_{1j,a}(\bfr) \}
\oominus   \{ \sum_{ij} \rhoij F^*_{2i,a}(\bfr)F_{2j,a}(\bfr) \}. \label{eq:n}
\end{eqnarray}
\end{widetext}
%$\sum_{ij} \rhoij F_i^*(\bfr) F_j(\bfr')$. As we see in
%\refsec{sec:augment}, this behaves as the one-body density
%matrix only when $\bfr \approx \bfr'$.
We can calculate the Coulomb interaction from $\calR[n]$. 
However, to reduce the computational effort, 
we will also make the Coulomb interaction into the separable form as
seen in \refsec{sec:coulomb}, with the help of multipole technique
due to Weinert \cite{weinert81}.
In \refsec{sec:multi} and \refsec{sec:frozencore}, we give some
preparations to define the Coulomb interaction in \refsec{sec:coulomb}.

The total energy should be given as a functional of eigenfunctions in
the first-principle calculations, not just as a functional of
coefficients $\{\alpha^j_p\}$.  This is important in some cases. For
example, it is necessary to know how the change in the basis set affects
the total energy when we calculate atomic forces. These are related to
the so-called Pulay terms \cite{pulay69}.

------------------------\\
浅いコアについてはPZを指定することでloをもちいて扱う
(注:デフォルトのPをセミコアにするときは、大きい整数値をもつPを指定する
必要あり。たとえばFeだとデフォルトで3dがvaleceなので
これをセミコア扱いして、「PZ=0,0,3.8 P=0,0,4.2」とすれば3d,4dをvalenceに
入れることができる。ただ、4dはある程度はPWでカバーできる。(実際こういう
設定を使うこともあるが、その有効性はもうすこし検証する必要あり)。

{\small 
ほかにもPZを13.8などとして、局在MTOにそのハンケル関数のtailを付け加える
方法もある。ただ数値的安定性などの点から現在はあまり利用していない。
注:このときには、semi-coreのエネルギーlevelで、MT内でradial schr\"odinger eq.を解く。
そして、それをMT外部へsmooth Hankel関数に接続した局在化基底をもちいる。
この処方箋は基本的には、通常のvalence電子に対してもちいるMTOとおなじであるが、
この場合envelope関数のsmooth Hankelを指定するパラメーターは自動最適化される。
このsemi-coreをつかうには、ctrlファイルではPZ=12.5などとして+10でいれる。
この場合にもlocal orbitalと言ったりするが用語として適切ではない;通常のLAPWで
のlocal orbitalは、MT内に拘束されたものであるので混同をまねく。
}

これ以外の取扱いでは、lmfではFrozen core近似で扱う(LFOCA=1)のが基本である。
これで、コアは、上述のように3成分表示で表現されることになる。
浅いコアはlocal orbitalで扱わざるをえないが、そうでなければ、
LFOCA=1のほうがよい。深いコアでMT内に十分に局在してるものなら、LFOCA=1でもLFOCA=0でもかまわない。
LFOCA=0は、MT内に閉じ込めた条件である(無理にMT内に局在させて解く)。
しかし、もし、NMRなどでコア位置での電場などを真面目にときたいのなら、LFOCA=0の方
がいいかもしれないし、そもそもAkaiKKRのradial schoredinge方程式を解くルーチンを組み
込んだ方がいいかもしれない(あるいはlmfに組み込まれているlocpot.F elfigr
でOKなのかもしれない。確認必要。)
全エネルギーに対してコアは、「コアの運動エネルギー」と「コアの電子密度」
を通じて寄与する。
Frozen core近似では、孤立原子の計算によって、それらを決定して用いる。
とくに、電子密度に関しては、孤立原子に関して得たものを単純に重ね合わせて固体
中のコア電子密度とする。(LFOCA=0の近似では、MT内で解いて決定する)。

%%%%%%%%%%%%%%%%%%%%%%%%%%%%%%%%%%%%%%%%%%%%%%%%%%%%%%%%%%%%%
\subsection{multipole transformation}
\label{sec:multi} In order to define Coulomb interaction in
Sec.\ref{sec:coulomb}, we introduce the multipole transformation
($\MM$-transformation) for the the 3-component functions.  This
corresponds to the compensation charges in Ref.\onlinecite{PAW}.

Before defining the $\MM$-transformation, we define the gaussian
projection $\GG_a \left[f(\bfr)\right]$ as follows.  The projection
$\GG_a \left[f(\bfr)\right]$ is defined for the function $f(\bfr)$ for
$|\bfr|\leq R_a$ as
\begin{eqnarray}
&&\GG_a \left[f(\bfr)\right]
=\sum_L Q_{aL}[f] G_{aL}(\bfr), \label{eq:gdef} \\
&&G_{aL}(\bfr)= \frac{1}{N_{aL}} \exp\left(-\left(\frac{r}{\RGSa}\right)^2\right)
Y_L(\hat{\bfr}),\label{eq:gl}
\end{eqnarray}
where $Q_{aL}[f]=\intaa \YY(\bfr) f(\bfr) d^3r$
gives the $L$-th multipole moment of $f(\bfr)$.
$N_{aL}$ is a normalization factor
so that $G_{aL}(\bfr)$ has a normalized multipole moment.
%Ritht? $N_{aL}=1/(\sqrt{\pi}RGS,a)^{3/2}}$
$\RGSa$ in \req{eq:gl} is chosen small enough so that $G_{aL}(\bfr)$ is
negligible for $|\bfr|\geq R_a$ (See Eq.(25) in
Ref.\onlinecite{lmfchap}). This $\GG_a \left[f(\bfr)\right]$ is a
superposition of gaussians $G_{aL}(\bfr)$ with keeping the multipole
moments of $f(\bfr)$.  We can take rather small $\RGSa$ without loss of
numerical accuracy; it is possible to take a limit $\RGSa \to 0$ because
quantities involved in $G_{aL}(\bfr)$ are evaluated analytically or
numerically accurately on a dense radial mesh.

We now define $\MM$-transformation for
3-component density $n= n_0 \ooplus n_1 \oominus n_2$ as
\begin{widetext}
\begin{eqnarray}
&&\MM[n]=  n_0(\bfr) \nonumber \\
&&+ \sum_{a,\bfT,L} Q_{aL}[n_{1,a}\!-\!n_{2,a}]G_{aL}(\bfr-\bfR_a-\bfT) 
\ooplus n_{1} \oominus 
\{n_{2,a}(\bfr) + \sum_L Q_{aL}[n_{1,a}\!-\!n_{2,a}]G_{aL}(\bfr) \}. \label{eq:multi}
\end{eqnarray}
\end{widetext}
Thus $\MM[n]$ adds the same gaussians to both of the 0th and 2nd
components.  $\bfT$ is the translational vectors of $\Omega$.  With this
transformation, the multipole moments of the 1st and 2nd components become
the same.  Note that the $\MM$-transformation is not a physically
meaningful transformation because $\calR[\MM[n]]=\calR[n]$.  With this
transformation, interstitial electrostatic potential calculated from the
0th component of \req{eq:multi} should be the same as that calculated
from $\calR[n]$.\\

------------\\
*将来的に「$E_{\rm es}$を多重極変換せずに直接にすべての項をとって評価する」
という改良を考えていいんじゃないかと思う。多重極変換などが不要でむしろ簡
単になる。逆にいえば、「現在のimplementationにおいて、
(一見きれいな風には見えるが)なぜ、多重極変換にこだわって$E_{\rm es}$
を評価する必要があるのか?」という疑問がある。メリットがあるかどうか検討必要。


%%%%%%%%%%%%%%%%%%%%%%%%%%%%%%%%%%%%%%%%%%%%%%%%%%%%%%%%%%%%%%%%%
\subsection{Coulomb interaction}
\label{sec:coulomb} In principle, we can define the Coulomb interaction
between $n(\bfr)=n_0\ooplus n_1 \oominus n_2$ and $m(\bfr)=m_0 \ooplus
m_1 \oominus m_2$ from the densities $\calR[n]$ and $\calR[m]$.  We can
use $\calR[\bar{n}]$ instead of $\calR[n]$ where $\bar{n}=\MM[n]$
satisfies $\calR[\bar{n}]=\calR[n]$, and $\calR[\bar{m}]$ as well.  Thus
the Coulomb interaction $\left(n|v|m\right)_{\rm original}$ is given as
\begin{eqnarray}
\left(n|v|m\right)_{\rm original}=\sum_{\bfT} \int_\Omega d^3r d^3r' 
\calR[\bar{n}(\bfr)] v(\bfr-\bfr'+\bfT) \calR[\bar{m}(\bfr')].
\label{eq:coulomb0}
\end{eqnarray}
Here $v(\bfr)=e^2/|\bfr|$; $\sum_{\bfT}$ implicitly includes the
division by number of cells. Equation~(\ref{eq:coulomb0}) can not be
easily evaluated because $v(\bfr-\bfr'+\bfT)$ contains the cross terms
which connect the 0th component with other components.

Thus we use an approximation 
\begin{eqnarray}
\left(n|v|m\right)\equiv\MM[n] \cdot v \cdot \MM[m] = \bar{n} \cdot v \cdot \bar{m}, \label{eq:coulomb}
\end{eqnarray}
instead of $\req{eq:coulomb0}$, where dot operator for the 3-component
functions is given as
\begin{eqnarray}
&&\bar{n} \cdot v \cdot \bar{m} \equiv
\bar{n}_0 \bullet v \bullet \bar{m}_0 + \bar{n}_{1} \circ v \circ
\bar{m}_{1} - \bar{n}_{2} \circ v \circ \bar{m}_{2}\\
&&\bar{n}_0 \bullet v \bullet \bar{m}_0 \equiv \sum_{\bfT} \int_\Omega d^3r d^3r' \bar{n}_0(\bfr)v(\bfr-\bfr'+\bfT) \bar{m}_0(\bfr'), \label{eq:n0vn0}\\
&&\bar{n}_1 \circ v \circ \bar{m}_1 \equiv 
\sum_a \inta d^3r \intad d^3r' \bar{n}_{1,a}(\bfr)v(\bfr-\bfr') \bar{m}_{1,a}(\bfr'), \label{eq:n1vn1}\\
&&\bar{n}_2 \circ v \circ \bar{m}_2 \equiv \sum_a \inta d^3r \intad d^3r' \bar{n}_{2,a}(\bfr)v(\bfr-\bfr') \bar{m}_{2,a}(\bfr'). \label{eq:n2vn2}
\end{eqnarray}
Note that $X \bullet Y$ means integral over $\Omega$, whereas $X \circ
Y$ means integrals within MTs.

Let us evaluate the difference between \req{eq:coulomb0} and \req{eq:coulomb}.
This can be evaluated with the identity in Appendix \ref{sec:zeroonetwo} as
\begin{eqnarray}
&&\left(n|v|m\right)_{\rm original}-\left(n|v|m\right)= 
\sum_a \inta d^3r \intad d^3r'\Big(
\left( \bar{n}_0(\bfr) \!-\! \bar{n}_2(\bfr) \right) 
v(\bfr\!-\!\bfr') \left( \bar{m}_1(\bfr')\!-\!\bar{m}_2(\bfr') \right)  \nonumber \\
&&+\left( \bar{n}_1(\bfr)\!-\!\bar{n}_2(\bfr) \right) 
v(\bfr\!-\!\bfr') \left( \bar{m}_0(\bfr')\!-\!\bar{m}_2(\bfr') \right)\Big).
\label{eq:coulombdiff}
\end{eqnarray}
This is essentially the same with Eq.(13) in Ref.\onlinecite{kresse99}.  In
\req{eq:coulombdiff}, the difference consists of contributions from MT
sites without terms connecting different MT sites. This is because
$\bar{n}_{1,a}(\bfr)$ and $\bar{n}_{2,a}(\bfr)$ have the same multipole
moments.  Since $\bar{n}_0(\bfr') - \bar{n}_2(\bfr)$ is high-$l$ or
highly oscillating part, and $\bar{n}_{1,a}(\bfr)-\bar{n}_{2,a}(\bfr)$
has zero multipole moments and zero at MT boundaries, we expect that the
separable form of \req{eq:coulomb} should be justified. We can check
this with changing the truncation parameters $l_{{\rm max},a}$ and 
$k_{{\rm max},a}$.

From \req{eq:coulomb}, we have the expression of the Coulomb
interaction as
\begin{eqnarray}
\left(F^*_i F_j|v|F^*_{i'}F_{j'}\right) = \MM[F^*_iF_j] \cdot v \cdot \MM[F^*_{i'}F_{j'}].
\label{eq:ffvff}
\end{eqnarray}
Here $F^*_i F_j$ is the diagonal product defined in \req{fnproduct} at
$\bfr=\bfr'$.  In calculations such as arising in the $GW$
approximations \cite{kotani07a}, we have to evaluate this as accurately
as possible so that the exchange-pair cancellation is kept well.


%%%%%%%%%%%%%%%%%%%%%%%%%%%%%%%%%%%%%%%%%%%%%%%%%%%%%%%%%%%%%%%%%%%%%%%%%%%%
\subsection{Frozen core approximation}
\label{sec:frozencore} We often need to treat spillout of the core
density outside of MTs explicitly. Then we use the frozen core
approximation; the charge density due to the cores are evaluated by a
superposition of rigid cores as follows \cite{lmfchap}.

First, we perform a self-consistent atomic calculation under the
spherical approximation without a spin polarization to obtain its core
density $\nc_a(\bfr)$. Then we make a fitted density $\nc_{{\rm
sH},a}(\bfr)$ given by a linear combination of several \smh\ functions
so that $\nc_{{\rm sH},a}(\bfr)$ reproduces $\nc_{a}(\bfr)$ for
$|\bfr|>R_a$ within a numerical accuracy. Since $\nc_{{\rm H},a}(\bfr)$
are analytic and smooth at their centers, we can treat them numerically
accurately (we can use other kinds of analytic functions such as
gaussians instead of \smh\ functions).\\

\ \\
Frozen core近似、LFOCA=1の場合,
まず原子のコアをlmfaで孤立原子において計算しておく。これをMT半径内でスムーズ
化したものを$n^{\rm c}_{\rm sH}(\bfr)$とする(電荷の規格化がなされてい
る必要はない)。この$n^{\rm c}_{\rm sH}(\bfr)$は、MT外においては、孤
立原子で計算したコアの電荷密度を再現するものである。
これを用いて,各原子に対して、コアの電子密度$n^{\rm c}(\bfr)$
を以下の3成分表示の形であたえておく。サイトのindex $a$は省略している。
\begin{eqnarray}
n^{\rm c}(\bfr)= 
n^{\rm c}_{\rm sH}(\bfr)
+ n^{\rm c}(\bfr)- n^{\rm c}_{\rm sH}(\bfr)
\label{eq:nc}
\end{eqnarray}
この右辺の各項が、第0,1,2成分である。第0成分がスムーズ化した$n^{\rm c}_{\rm sH}(\bfr)$
である。また、第1、2成分は当然、MT内のみの成分である。corprm.fで計算されるcofh
が、$n^{\rm c}_{\rm sH}(\bfr)$に対する重みであり,MTの外の電子密度が
正確に再現されるようになっているようである(要確認)。
lmfにおいては、この$n^{\rm c}_{\rm sH}(\bfr)$は
smooth Hankelでフィットされたものを用いており全空間にひろがっている。
(そもそもは、smooth Hakelは波動関数をフィットするためのものであって電子
密度をフィットするものではなかったが解析性などを考えこれを用いている)。
smooth Hankelの解析性のために、静電相互作用などの計算が容易になる。
フィットの係数とsmooth Hankelを指定するパラメーター
は適当に決定されている;fp/locpot.Fのlocpt2のドキュメントなど参
照.cofg,cofh,ceh,rfocなど。また詳細を書くこと。wiki?。\\
  
Thus we have the expression of all the core electron density with adding
contribution from nucleus $-Z_a\delta(\bfr)$:
\begin{eqnarray}
\nzc = \sum_{a,\bfT} \nc_{{\rm sH},a}(\bfr-\bfR_a-\bfT) \ooplus
\{\nc_{a}(\bfr)-Z_a\delta(\bfr) \} \oominus \{\nc_{{\rm sH},a}(\bfr) \}.
\label{eq:nzc}
\end{eqnarray}
Applying the $\MM$-transformation to $\nzc$ gives
\begin{eqnarray}
\MM[\nzc]
&=& \sum_{a,\bfT} 
\left( \nc_{{\rm sH},a}(\bfr-\bfR_a-\bfT)+\sum_L Q_{aL}^{\rm Zc} G_{aL}(\bfr-\bfR_a-\bfT) \right)\ooplus
\{\nc_{a}(\bfr)-Z_a\delta(\bfr)\} \nonumber \\
&& \oominus \{ \nc_{{\rm sH},a}(\bfr)+\sum_L Q_{aL}^{\rm Zc}G_{aL}(\bfr)\},\label{eq:nc} \\
Q^{\rm Zc}_{aL} &=& Q_{aL}[\nzc_1-\nzc_2] =
Q_{aL}[\nc_{a}(\bfr)-Z_a\delta(\bfr)-\nc_{{\rm sH},a}(\bfr)] \label{eq:qalnzc}.
\end{eqnarray}

\ \\

MT内にコアが局在するLFOCA=0の場合においては、
上の式において、$n^{\rm c}_{\rm sH}$の項を除いたものとなる。


%%%%%%%%%%%%%%%%%%%%%%%%%%%%%%%%%%%%%%%%%%%%%%%%%%%%%%%5
\subsection{total energy in density functional}
\label{sec:total} Let us give the total energy $E_{\rm total}$ for the
DFT, and construct the Kohn-Sham equation from it. With the kinetic
energy $\ek=\frac{1}{2m_e} \sum_{ij} \rhoij \langle \nabla F_i| \nabla
F_{j} \rangle$ from Eq.(\ref{eq:kin}), the total energy is given as:
\begin{eqnarray}
E_{\rm total}=\ekcore+ \ek+E_{\rm es}+E_{\rm xc},
\label{eq:etot}
\end{eqnarray} 
where $\ekcore$ is the kinetic energy of frozen cores as a constant.
$E_{\rm es}$ and $E_{\rm xc}$ are electrostatic and exchange-correlation
energies, respectively.
$E_{\rm es}$ is given as the electrostatic energy for the
total density $\nzcv =\nzc +n $, which are given in
Eqs(\ref{eq:n},\ref{eq:nzc}). 

Based on the definition \req{eq:ffvff}, we have
%=\nzc(\bfr) +\sum_{ij} \rhoij F_i^*(\bfr) F_j(\bfr)$ as:
\begin{eqnarray}
E_{\rm es}= \frac{1}{2} (\nzcv|v|\nzcv)=\frac{1}{2} \MM[\nzcv] \cdot v \cdot \MM[\nzcv],
\label{eq:es}
\end{eqnarray}
where a constant due to the self-interaction of nucleus is implicitly
removed. 
Components of $\barnzcv(\bfr)=\MM[\nzcv]$ are given as
%contains smHankel from  $\nzc(\bfr)$, Gaussians $G_{aL}$ due to ${\cal M}$
%transformation, and the density on real space mesh due to $F^*_{0i}(\bfr)F_{0j}(\bfr)$:
\begin{eqnarray}
&&\barnzcv_0(\bfr)= \nzc_{0}(\bfr)
%\sum_a \nc_{{\rm sH},a}(\bfr-\bfR_a) 
+ \sum_{a,L,\bfT} 
(Q_{aL}^{\rm Zc}+Q^{\rm v}_{aL})G_{aL}(\bfr-\bfR_a-\bfT)+ n_0(\bfr),
\label{eq:barn0zcv}\\
&&\barnzcv_{1,a}(\bfr)=\nzc_{1,a}(\bfr) +n_{1,a}(\bfr), \label{eq:barn1zcv}\\
&&\barnzcv_{2,a}(\bfr)=\nzc_{2,a}(\bfr) 
+\sum_L (Q_{aL}^{\rm Zc}+Q^{\rm v}_{aL})G_{aL}(\bfr)+n_{2,a}(\bfr),\label{eq:barn2zcv}
\end{eqnarray}
where $Q^{\rm v}_{aL}=Q_{aL}[n_{1,a}-n_{2,a}]$.  We expand
$F^*_{0i}(\bfr)F_{0j}(\bfr)$ of $n_0$ in $\{e^{i \bfG \bfr}\}$ (to
obtain coefficients, $F^*_{0i}(\bfr)F_{0j}(\bfr)$ is tabulated on a
real-space mesh, then it is Fourier transformed).  The cutoff on $\bfG$
is specified by $\EMAXm$.  Then the 0th components in \req{eq:barn0zcv}
is represented by sum of analytic functions. Thus we can finally
calculate $\frac{1}{2} \barnzcv_0(\bfr) \bullet v \bullet
\barnzcv_0(\bfr)$ in $E_{\rm es}$.  Terms between gaussians located at
different MT sites are evaluated with the Ewald sum treatment.  The
terms related to MTs in $E_{\rm es}$ is $ \frac{1}{2} \barnzcv_1 \circ
\RR \circ \barnzcv_1 - \frac{1}{2} \barnzcv_2 \circ \RR \circ
\barnzcv_2$, which is calculated on a radial mesh accurately.

The exchange correlation term can be defined as
\begin{eqnarray}
E_{\rm xc}[\nzcv] = E_{\rm xc}[\nzcv_0] 
+  \sum_a E_{\rm xc}[\nzcv_{1,a}]  
-  \sum_a E_{\rm xc}[\nzcv_{2,a}].
\label{eq:exc}
\end{eqnarray}
The functional derivatives of $E_{\rm xc}[\nzcv]$
with respect to each component of $\nzcv$ gives
\begin{eqnarray}
v^{\rm xc} = v^{\rm xc}_0(\bfr)
\ooplus  \{v^{\rm xc}_{1,a}(\bfr)  \}
\oominus \{v^{\rm xc}_{2,a}(\bfr)  \}.
\label{eq:vxc}
\end{eqnarray}
%%%%% Need to be improved if we write something here...
%% We evaluate $E_{\rm xc}[\nzcv_0]$ on real-space mesh.
%% Though $\nzcv_0$ is expanded in the $G$ vector, $\nzcv_0$
%% keeps translational symmetry. However,
%% we use real-space meshing in the evaluation of 
%% $E_{\rm xc}[\nzcv_0]$, this can slightly break the translational
%% symmetry. In anyway, there remains a future problem that how to avoid dense
%% real-space mesh for the calculations;
%% problem is not in the absolute value of the total energy, but in its systematic error
%% when we move atomic positions $\bfR_a$ relative to the grids of real-space mesh.
%% Note that $E_{\rm es}$ do not have this problem since all quantities 
%% are analytically treated for a given cutoff $\EMAXm$.

To determine the ground state, 
$E_{\rm total}$ should be minimized under the orthogonality of eigenfunctions with
the constraint (A') and (B).
This ends up with 
$\delta \psi^*_p \cdot (H - \epsilon_p) \cdot \psi_p =0$ for
the variation $\delta \psi^*_p$ which satisfy (A') and (B). 
Here the operator $H=T+V$ is given as
\begin{eqnarray}
&&T=\frac{-\nabla^2 }{2m_e} \oplus \left\{\frac{-\nabla^2 }{2m_e}\right\} 
\ominus \left\{\frac{-\nabla^2 }{2m_e}\right\} \\
&&V= 
\barnzcv_0 \bullet v \bullet + v^{\rm xc}_0
\ooplus
\left\{ \sum_L
{\cal Q}^{\rm v}_{aL} \YY_L(\bfr)
+ \barnzcv_{1,a} \circ \RR \circ + v^{\rm xc}_{1,a} 
\right\} \nonumber \\
&&\oominus
\left\{ \sum_L
{\cal Q}^{\rm v}_{aL} \YY_L(\bfr)
+ \barnzcv_{2,a} \circ \RR \circ + v^{\rm xc}_{2,a} 
\right\}, \label{eq:v} \\
&&{\cal Q}^{\rm v}_{aL}\equiv \frac{\partial E_{\rm es}}{\partial {Q}^{\rm v}_{aL}} =
\barnzcv_0 \bullet v \bullet G_{aL}(\bfr'-\bfR_a)
-\barnzcv_{2,a} \circ \RR \circ G_{aL}(\bfr'), \label{eq:calqdef}
\end{eqnarray}
where $\bar{n}_0 \bullet v \bullet$ means an integral on a variable, resulting a function of $\bfr$.

When a basis set $\{F_j(\bfr)\}$ satisfying (A') and (B) are fixed, we
just need to consider variation with respect to $\alpha_{p}^{i*}$ in
\req{eq:eig}. Then we have
\begin{eqnarray}
\sum_j (H_{ij} -\epsilon_p O_{ij}) \alpha_p^j =0,
\label{eq:eigenp}
\end{eqnarray}
where $H_{ij}= \langle F_i| H |F_{j} \rangle = 
\langle F_i| \frac{-\Delta}{2m} + V |F_{j} \rangle =T_{ij}+V_{ij}$.
$V_{ij}=\langle F_i|V|F_{j} \rangle=V \cdot F^*_i F_j$. 
%Here an
%operator in 3-component 
%$V=\frac{\delta E}{\delta n}=V_0\ooplus V_1 \oominus V_2$ is given as
%where terms including $\dot{Q}^{\rm v}_{aL}$ comes from the derivative
%through $Q^{\rm v}_{aL}$ in \req{eq:barn0zcv} and in \req{eq:barn2zcv}.
Then the total energy minimization results in the eigenvalue problem.
The matrix elements $O_{ij},T_{ij}$
and $V_{ij}$ are given in Appendix \ref{onsitematrix}.

The formula to evaluate atomic forces are given in Appendix
\ref{sec:force}. It is directly evaluated from the variation on the
total energy. This procedure is considerably simplified than that given
in Refs.\cite{lmfchap,molforce}. In addition, we can show that the
atomic forces are not dependent on the real-space mesh relative to the
atomic positions as shown in Appendix \ref{sec:realmesh}. This is
important to allow relaxations of atomic positions and to perform
first-principle molecular dynamics.\\

----\\
$E_{\rm es}$の第0項に関しては、mkpot.F内のsmves.Fで評価している。
この中で、Gaussian$\times$smooth partの積分やGaussian$\times$Gaussianの積分が出て
くる。後者については、off-siteの部分に関しては、単純に多重極のエバルト和で評価できる。
第1項第2項に関しては、locpot.Fで評価している。

%%%%%%%%%%%%%%%%%%%%%%%%%%%%%%%%%%%%%%%%%%%%%%%%%%%%%%%%%%%%5
\section{PMT method}
\label{sec:pmtmethod} Let us give the PMT method based on the
3-component formalism in \refsec{sec:formalism}. Based on it, we need to
specify a basis set $\{F_i\}$.  In the PMT, $\{F_i\}$ is classified into
three kinds of subsets as follows:
\begin{itemize}
\item[(a)] APW. We augment the PW in the manner as will be shown later.

\item[(b)] MTO. We augment the atom-centered \smh\ functions. 
           %Its centers are at $\bfR_a$. 
           %We take Bloch sum of \smh\ to satisfy crystal symmetry.
           %The head part is augmented analytically (thus perfectly
           %augmented);  the tail part is
	   %augmented in the manner of \refsec{sec:augment}. 
           %As we explain in \ref{sec:result}, we usually use two MTOs for
           %each $aL$ channel, although we use three MTOs in some cases.
           %, except
           %some cases. For $sp$ channel for rare gas and $d$ channel
           %for Cu through As, we use three MTOs.

\item[(c)] Local orbital (Lo). It is origiannly introduced for LAPW in Ref.\onlinecite{lo}.
   We use this to represent some degree of freedom in MTs,
   such as semicore states. The envelope function of Lo is zero overall.

\end{itemize}
%we use three kinds of 
%The PMT method uses wo kinds of augmented basis, the APW and the MTO
%for valence electrons. These are made of smooth functions 
%called as ``envelope functions'' through an augmentation procedure.
%xxxxxxxx
%For the envelope functions in the PMT method, we use the PW for the APW, and
%the smooth Hankel (\smh) function for the MTO. In addition, it uses local orbitals.
%%%%%%%%%%%%%%%%%%%%%%%%%%%%%%%%%%%%%%%%%%%%%%%%%%%%%%%%%%%%% 
%\subsection{one-center expansion}
%\label{sec:onecenter}

(a),(b)はエンベロープ関数をaugmentationすることで作られる。
エンベロープ関数として、APWについてはPWを、MTOについては、
\smh をもちいる。The \smh \ は以下の、\req{eq:defh0}のように
Methfesselによって導入された関数である \cite{lmfchap,Bott98}.
(b)と(c)は原子を中心とする局在基底であり、周期境界条件を考えると、
Bloch和をとって基底関数とすることになる。

MTO基底に関しては、各$L$あたり2個程度用いれば、APWのカットオフを3-4Ry程
度にとることができる;これはかなり低いエネルギーである。このときMTOの基
底は、EH=-1Ry,EH=-2Ryとかなり大きなダンピングのものをつかうことができる。
これはAPWを用いることで得られるメリットである。

また、smooth Hankelの曲げの自由度をエネルギー最小化に利用するのは
かなり面倒であり(おそらく無意味)、RSMH、RSMH2は0.5*R程度と十分に小さ
くとる、という原則を守ったほうがよいようである---結局、MTの外でハンケル的に
振る舞うようにしてやるのがベストな選択ということである。

このような基底で、2原子分子の結合エネルギーで1Kcal/mol程度の精度が期待で
きる(遷移金属などでは絶対値はそこまでしっかり収束するわけではない)。
単純には、MT近傍はMTO記述し、MTとMTをつなぐ中間領域は、APWで補正してやる、
というpictureである。APWの重ね合わせで局所化させたワニエ的関数をつくる
ことも考え得るが、これは結局は「floating orbital」の手法
(MT半径ゼロの正エネルギーのMTOを考える手法)と似たものになる
(負エネルギーのものに関しては実装されている)。
緩やかな振動が長距離に及ぶ(PWに関してsin(kr)/rで1/rで落ちる)
のでどれだけメリットがあるか不明である。



\subsection{sm hankel function and augmentation}
\label{sec:smh}
The \smh\ function, as the envelop function of MTO, is first introduced
by Methfessel \cite{lmfchap,Bott98}. The spherical \smh\ function
$h_0(\bfr)$ (for $l=0$) is defined by the Helmholtz equation with a
gaussian source term $g_0(\bfr) = C \exp(-r^2/\RSM^2)$ (see Eq.(5) in
Ref.\onlinecite{lmfchap}) instead of $\delta$-function;
\begin{equation}
(\nabla^2+\epsilon)h_0(\bfr) = -4 \pi g_0(\bfr),
\label{eq:defh0}
\end{equation}
where $C=1/(\sqrt{\pi} \RSM)^3$ is the normalization constant.
$\epsilon=-\kappa^2$ is the negative energy to specify the asymptotic
damping behavior of $h_0(\bfr)$.  At the limit $\RSM \to 0$ where
$g_0(\bfr)$ becomes $\delta$-function (as a point charge), $h_0(\bfr)$
becomes to the Hankel function $h_0(\bfr)=\exp(-kappa r)/r$.  Since the
source term is smeared with the radius $\RSM$, we have no divergent
behavior at $r=0$ anymore; the \smh\ bends over at $\sim\RSM$ (See Fig.1
in Ref.\onlinecite{lmfchap}).  From $h_0(\bfr)$, we can make
$h_L(\bfr)\equiv\YY_L(-\nabla) h_0(\bfr)$ for any $L$ with the spherical
polynomial $\YY_L(\bfr)=r^l Y_L(\hat{\bfr})$.  $Y_L(\hat{\bfr})$ is the
real spherical harmonics, where $\hat{\bfr}$ is the normalized
$\bfr$. $\YY_L(-\nabla)$ means to substitute $\bfr$ in $\YY_L(\bfr)$
with $-\nabla$. See Ref.\onlinecite{Bott98} for details.

For the augmentation of the PW, that is, to determine the 2nd component
from PW of 0th component, we expand the PW within the MTs into the
Laguerre polynomial \cite{pmt1}.  Any function $f(\bfr)$ (PW in this
case) is expanded within a MT $|\bfr-\bfR_a|\leq R_a$ as
\begin{eqnarray}
&&f(\bfr)= \sum_{k,l} \CakL[f] \PakL(\bfr-\bfR_a), \label{onecenter1}\\
&&\PakL(\bfr)=\pakl(r) Y_L(\hbfr) \label{onecenter2},
\end{eqnarray}
where $k=0,1,2,...$ denotes the order of a polynomials $\pakl(r)$.  In
the case that $f(\bfr)$ is a PW, the coefficients for the function
$\CakL[f]$ are given analytically \cite{Bott98}.

When we use \smh\ centered at $\bfR_a$ as an envelope function
$f(\bfr)$, 
we have head part, which is $f(\bfr)=h_L(\bfr-\bfR_a)$ for
$|\bfr-\bfR_a| \leq R_a$, and tail part, which is in other MT
sites $|\bfr-\bfR_{a'}| \leq R_{a'}$.  As for the tail part, we use the
expansion of \req{onecenter1} as in the case of PW. On the other hand, we
use the head part as it is \cite{privatemark1};
% This is because it is not easy to expand the
% head part in such a expansion because of too steep nature of the funcion.
this can be taken into account in the formalism 
if the set $\{\PakL(\bfr) \}$ contains not only the Laguerre
polynomials but also $h_L(\bfr)$ as its members.

% In the followings, we apply this expansion not only to the PW, 
% but also to the \smh\ centered at $\bfR_a$.
% However, it is not easy to apply this expansion to the central MT of
% the \smh, that is, 
% this part is called as the head of the \smh.
% This is because it is too steep to be expanded 
% accurately with finite cutoffs in this series in practice.

%Here (b) and (c) are atom-centered localized orbitals, for which we
%take the Bloch sum to recover translational symmetry. 

After specifying $\{\PakL(\bfr) \}$, we can determine corresponding
$\{\wPakL(\bfr) \}$ as a linear combination of
$\phi_{al}(r)Y_L(\hat{\bfr})$ and $\phidot_{al}(r)Y_L(\hat{\bfr})$,
where partial waves $\phi_{al}(r)$ and its energy derivatives
$\phidot_{al}(r)$ are given as the solutions of the radial Schr\"odinger
equation for the spherically-averaged potential of $V_{1,a}$ in
\req{eq:v}, where energies $E_{al}$ to solve the equation are given as
the center of gravities of the occupied states of the partial density of
states of the $al$ component; thus 
$\phi(r)$ and $\phidot(r)$ are not with the subscripts $aL$ but with $al$.
This prescription to determine $\{\wPakL(\bfr) \}$ can be taken as a
quasi-minimization procedure, from the view of total energy
minimization.\\

----\\
NULL基底が生じること:\\
一つ注意すべき点は、このphi, phidotによるaugmentationはnull基底を生じ
せしめることである。すなわち、PAWとちがって、$k$の自由度
(\req{onecenter1}におけるradial関数の数)は、phi,phidotの数の
2よりも大きいので(通常(k=0,1,...,KMXA=5)と6程度に取る)、
異なる$\CakL[f]$をもつ$f$をaugmentした結果、同じphi,phidotの係数を得るこ
ともありえる。差をとることを考えれば、augmentしてphi,phidotの係数がゼロ
になることがおきたとしても、それに対応する$f$がゼロではないという基底が
まざりうるということである。これはヌルベクトル
である;すなわち、第0成分=第2成分がMT内部に局在して第1成分がゼロという
関数が基底に含まれることになる(これは物理的に意味をもたない)。
これが「overcompleteness」が起こる要因である(のひとつである)と思われる。\\

また、内積の正定値性の問題がある。
3-component spaceの内積は、\req{eq:norm}で与えられているという
問題がある。この内積は負符号部分を含むのでその固有値は-1から1までとなる。
きちんと(A)条件(すなわち第0成分のMT内成分が第2成分と一致する)が満た
されていない場合、一般にはたとえば、(1,1,0,5)+(-1,-1,0.5)のノルムが
マイナスになるというようなことが起こり得る(これは3成分を単純に数字で表して考えてみた例)。
pwemaxを大きくするとき、kmxaをおおきくしないといけない、というような点に
この問題が現れているようにも思える。\\

実際の計算において、収束が破綻するとき第0成分=第2成分が数値的に
やたらと大きくなるという状況があった。これは、本質的にこれらの成分が不定
であるという点に問題があると思える。対角化を行うとき第2成分ができるだけ
小さくなるように、人為的に、ポテンシャルを加えてやるという工夫が考えられ
る。実際のところ、計算の破綻が起きやすいのは、LaやNaなどでMT半径を大きく取っ
た場合である。sp電子が低エネルギーの平面波で表せてしまうので、、ヌルベク
トルが見えてしまう。これを高いエネルギーのところに飛ばしてしまう必要があ
る(単純に自由度を引き抜くようなやり方ではどうも系統性をこわしてしまって
よろしくないようだ)。\\

\ \\


As for the $al$ with Lo, we need another partial wave $\phi^{\rm
Lo}_{al}(r)$ corresponding to Lo.  When the Lo is to describe a deeper
level, we can set the energy to solve the radial Schr\"odinger equation
$E^{\rm Lo}_{al}$ at the center of gravity; then we set $E_{al}$ at the
Fermi energy instead of the prescription in the previous paragraph.

The number of basis is simply specified by the cutoff energy of the APW
for (a). However, specification of MTOs (b) is not so simple.  
We use multiple MTOs for each $aL$ to reduce the number of basis with 
keeping the computational accuracy \cite{pmt1}.
Since $h_L(\bfr)$ as the envelope functions are specified 
by the parameters $\RSM$ and $\epsilon$, we have to specify them for
all MTOs. Ref.\onlinecite{lmfchap} discussed optimization of them so as
to minimize the total energy. However, as seen in figures in
Ref.\onlinecite{lmfchap}, such non-linear optimization is too complicated.
%In fact, most serious problem in the full-potential LMTO method is in the
%difficulty to optimize $\RSM$ and $\epsilon$.
Thus it is necessary to give a method to set the parameters 
in a simple manner as follows. 
As for $\RSM$, we can use a condition $\RSM= R_a/2$ for all MTOs. 
Then the envelope functions out side of MTs well coincide with
the usual Hankel function. 
Even with this simple setting of $\RSM$ without optimization,
numerical accuracy can be kept well; we can check the
convergence of calculations with the number of APWs.
We also see the dependence on $\epsilon$'s are rather small
in the PMT method. The dependence becomes less when we use larger
number of APWs; hence we do not need to stick to careful choice of
the parameter $\epsilon$.
Thus the serious problem of the full-potential LMTO method, ``how
to choose MTO parameters'' are essentially removed in the PMT
method. This is numerically detailed in the paper which gives results for
diatomic molecules from H$_2$ through Kr$_2$ \cite{kotanimol2011}.

We use one further approximation. In \req{eq:n}, we make angular-momentum cutoff.
Even though we have angular momentum component up to 
$2\times l_{{\rm max},a}$ in the 1st and 2nd components in \req{eq:n}, 
we drop components higher than $l_{{\rm max},a}$; 
it is meaningless to take them into account since we have already make 
truncations for eigenfunctions. Note that this does not affects $O_{ij}$
and $T_{ij}$ because only the special components determine them.\\

基底関数のインデックスの集合は$i\equiv\{\bfG,aLn\}$である。すなわち、
$i$は,$\bfG$か$aLn$のどちらかをとる。$\bfG$ は (a)をあらわし、$aLn$は (b)または
(c)をあらわす。ここで$n$は与えられた$aL$に対して
MTOの種類やloを指定するindexである.($n$が電子密度の記号と重なってしまっているが、
あとからこれは出てこないのでとりあえず問題ない。$a$は最初に書いてあるが
MTを指定するindex).それで、基底関数の集合は$\{F_i\}=\{F_{\bfq \bfG},F_{\bfq aLn}\}$と
書ける。APWの集合の指定に関しては、カットオフ$|\bfq+\bfG|<\EMAX$を用いる。

%%%%%%%%%%%%%%%%%%%%%%%%%%%%%%%%%%%%%%%%%%%%%%%%%%%%%%%%%%
\subsection{problems in the PMT method}
\label{sec:problems}
Let us examine three problems of the PMT methods, and ways to
manage them.

The first problem is the positive definiteness of $O_{ij}$.
Since the last term in \req{eq:norm} can give negative contribution,
there is a possibility that $O_{ij}$ can not be positive definite.
In principle, we can expect almost zero eigenvalues on the matrix
$\intaa d^3r \left(F^*_{0i}(\bfr)F_{0j}(\bfr)-F^*_{2i,a}(\bfr)F_{2j,a}(\bfr)\right)$ 
for all MTs if the truncation parameters are large enough. 
This guarantees the positive definiteness of $O_{ij}$.
In practice, we typically use $k_{{\rm max},a} \sim 5$ 
and $l_{{\rm max},a}\sim 4$; they
can give satisfactory results with keeping positive
definiteness of $O_{ij}$, as seen in Refs.\cite{pmt1,kotanimol2011}.

The second is the undefiniteness of the second component $\psi_{2p}$.
This is clear if (A) is satisfied; as $\psi_{2p}$ within MTs is not uniquely
determined since it is canceled completely by $\psi_{0p}$ within MTs.
However, since we use (A') in practice, this can cause numerical instability.
To illustrate this, let us consider a linear combination of basis functions
where only their 0th and 2nd components within MT are non zero.
This is a null vector which has no physical meanings; it gives zero when 
we apply Hamiltonian and Overlap matrix to it. This is a kind of ghost. 
Apparently, this occurs because the 3-component space is not a complete
metric space in the mathematical sense.
When we enlarge number of basis, this null vector can cause numerical problems.
It can be an origin of uncontrollable eigenvalue (e.g, 0 divided by 0), or
it can attach to some eigenfunctions and deform them easily.
In fact, we observed unconverged cases when the 2nd component of
electron density becomes too large. 
Within our current implementation of the PMT, we should use limited number
of basis so as to avoid this problem. However, in Refs.\cite{pmt1,kotanimol2011},
we can see enough stability on the total energy convergence before such problems occurs
when we increase the number of basis .

It will be possible to remove such undefiniteness in some manners.
For example, we can minimize the total energy with adding a 
fixing term $+\lambda \sum_p \intaa d^3r
\psi^*_{2p,a}(\bfr)(1-\tilde{P})\psi_{2p,a}(\bfr)$, 
where $\lambda$ is a Lagrange multiplier, $\tilde{P}$ is a projector to
the space spanned by some pseudo partial waves corresponding to true
atomic partial waves. If $\lambda$ is infinite,
2nd components are only spanned by the pseudo partial waves.
However, we should avoid a large $\lambda$ so as not to 
deteriorate the total energy minimization.

The third problem is the orthogonality to the cores. 
In the frozen core approximation in \refsec{sec:frozencore}, 
we take account of the spillout of the core electron density 
from MTs; this allows us to use a small MT radius.
However, when we use quite small MTs, we observed a problem of orthogonality
of wavefunctions to the cores, resulting in unconvergence. In such a case, 
we  need to introduce local orbitals to represent cores so as to keep
the orthogonality. It may be possible to enforce the orthogonality 
with a projector as described in Ref.\onlinecite{PAW}.


%%%%%%%%%%%%%%%%%%%%%%%%%%%%%%%%%%%%%%%%%%%%%%%%%%%%%%%5
\subsection{comparison with PAW}
\label{sec:comparison}
Here we will make a comparison of the PMT method with the PAW method
\cite{PAW,kresse99} based on the 3-component formalism.

In the PAW method, we perform the all-electron (AE) 
calculations for a spherical atom as a reference in advance.
Then the main problem is how to solve the one-body problem
for a given one-body potential $V(\bfr)$ in real space. 
As in \refsec{aug3}, the problem is translated into the problem in the 3-component space
for $V=V_0\oplus V_1 \ominus V_2$.
For simplicity, we omit the index $a$ in the followings.

The basis set in the PAW is given as follows.
We first prepare AE partial waves
$\{\phi_i(\bfr)\}$ (e.g, two for each $aL$ in Ref.\onlinecite{kresse99}),
as solutions of radial Sch\"odinger eq. for $V_1$ at some reference 
energies $\{\epsilon_i\}$ (in this section, the index $i$ is for the
partial wave). Then we set up corresponding
pseudo partial waves $\{\tilde{\phi}_i(\bfr)\}$.
The eigenfunction $\psi$ in the PAW can be represented 
in the 3-component space; for given 0th-component $\bar{\psi}$
(this is called as {\it pseudo wavefunction}), we have
$\psi$ with projectors $\{\tilde{p}_i\}$ as
\begin{eqnarray}
\psi=\bar{\psi}
\oplus \sum_i |\phi_i\rangle \langle \tilde{p}_i|\bar{\psi}\rangle
\ominus \sum_i |\tilde{\phi}_i \rangle \langle
\tilde{p}_i|\bar{\psi}\rangle.
\label{eq:psibar}
\end{eqnarray}
Here $\tilde{p}_i$ should satisfy
$\langle \tilde{p}_i|\tilde{\phi}_j \rangle=\delta_{ij}$.
The minimization of the total energy of the one-body problem 
$E= \sum_j^{\rm occupied} \psi_j^* \cdot (T+V) \cdot \psi_j$
with respect to $\bar{\psi}_j$ is given by
%\begin{widetext}
\begin{eqnarray}
&&\left( \frac{-\nabla^2}{2m}+V_0(\bfr)-\epsilon_j +
 \sum_{ii'}  
  |\tilde{p}_i \rangle  
   \left(dH_{ii'}-\epsilon_j dO_{ii'} \right)  
  \langle \tilde{p}_{i'}| 
\right) \bar{\psi}_j=0, \label{eq:seqpsibar}\\
&&dH_{ii'}= \langle \phi_i| \frac{-\nabla^2}{2m}+V_1|\phi_{i'}\rangle  
 -  \langle \tphi_i| \frac{-\nabla^2}{2m}+V_2|\tphi_{i'}\rangle \\
&&dO_{ii'}= \langle \phi_i|\phi_{i'}\rangle  
 -  \langle \tphi_i| \tphi_{i'}\rangle.
\end{eqnarray}
If we use infinite number of partial waves which makes a complete set, 
\req{eq:seqpsibar} reproduces the original one-body problem in real space.
% since 0th-components within MT
%completely cancels 2nd component because of 
%$1=\sum_{i}|\bar{\phi}_i\rangle \langle \tilde{p}_i|$;
%here we need to assume completness of $\{\phi_i\}$ also.

Let us consider a case where
$\psi_j=\bar{\psi}_j \oplus \phi_j\ominus\tphi_j$
is the solution of \req{eq:seqpsibar} with eigenvalue $\epsilon_j$,
where $\bar{\psi}_j$ within MT coincides with $\tphi_j$.
This is given by \req{eq:psibar} from $\bar{\psi}_j$.
When we make a truncation for the number of partial waves,
$\{\tilde{p}_i\}$ should satisfy 
\begin{eqnarray}
&& \left( \frac{-\nabla^2}{2m}+V_0(\bfr)-\epsilon_j \right)|\tphi_j \rangle
 + \sum_i |\tilde{p}_i \rangle  
   \left(dH_{ij}-\epsilon_j dO_{ij} \right)  =0, \label{eq:condproj}
\end{eqnarray}
%\end{widetext}
in order to satisfy \req{eq:seqpsibar}.
This determines $\{\tilde{p}_i\}$; this is one of the main idea in the PAW method. 
In practice, considering the numerical stability, 
we determine  $\tilde{p}_i$ so that \req{eq:condproj} is approximately
satisfied \cite{PAW}. 

Another important idea of the PAW is the introduction of the
pseudopotential. This is how to determine $V_0$ within MT ($=V_2$).
This is because the result strongly depends on the pseudopotential
when the number of partial waves are small. In principle, the pseudopotential 
should be determined so that $\bar{\psi}_j$ contain high energy part 
(high angular momentum $l$ or highly oscillating part) of the wavefunctions
which is missing in the 1st and 2nd components due to the 
truncation of the number of partial waves.

Note that the truncation can cause the ghost state problem in the PAW method.
To illustrate this, consider a case that $s$ wave part in MT is described
only by two partial waves $2s$ and $3s$. Then the PAW procedure maps $\bar{\psi}$ with zero node to 
$\psi$ with one node, $\bar{\psi}$ with one node to $\psi$ with two nodes. 
Problem is that $\bar{\psi}$ with two nodes, which is orthogonal to $\{\bar{\psi}_i\}$ for $2s$ and $3s$,
can not be mapped to $\psi$ with three nodes due to the truncation. 
Thus it is possible that such a function cause a ghost state;
we have to design the pseudopotential so that such $\bar{\psi}$ should be
kept to be at a high enough energy region (to push $\bar{\psi}$ high away from the
Fermi energy, it may be better to use relatively strongly repulsive pseudopotential). 
Ref.\cite{kresse99} claims that there is no
ghost state for all kinds of atoms. However, it is not easy
to check the convergence within the framework of the PAW method.

In the PAW method with PWs proposed in Ref.\cite{kresse99}, many PWs are required
compared with with LAPW. Roughly speaking, energy cutoff of PWs are
$\sim$15Ry in LAPW, and $\sim$30Ry in PAW \cite{filippi94,kresse99}.
This is because the PAW method, as is the case of pseudopotential methods,
needs to uniquely determine the pseudo partial waves (0th component) within MT.
This is in contrast with the LAPW (and the PMT) method, 
where 0th component within MT is irrelevant because 
the 2nd components have enough degree of freedom to well cancel its contribution.
However, with sacrificing the cutoff energy,
the PAW takes robust convergence that comes from the absence of the 
null vector problem discussed in \refsec{sec:problems}

As a theoretical possibility, we can imagine a method to 
use \smhs\ together with the PWs in the basis set
for the one-body problem in the PAW method.
However, it is not very clear whether it becomes 
a efficient method or not. To reduce the number of basis of PWs, it is
necessary to make the \smhs\ span high-energy parts of pseudo
wavefunctions. Thus we have to tailor \smh\ so that it fits to the pseudo
wavefunctions not only interstitial region, but also within MT. 
This can be not straightforward.


%%%%%%%%%%%%%%%%%%%%%%%%%%%%%%%%%%%%%%%%%%%%%%%%%%%%%%%%%%%%%%%%%%%%%%%%%%%%%%%%
%\appendix
%%%%%%%%%%%%%%%%%%%%%%%%%%%%%%%%%%%%%%%%%%%%%%%%%%%%%%%%%%%%
%\section{smooth Hankel function}
%\label{shankel}
\section{Other points in the PMT method}
\subsection{the error due to the 3-component formalism}
\label{sec:zeroonetwo}
To evaluate matrix element of a quasilocal operator
$X(\bfr,\bfr')$, we use separable form $0X 0'+1 X 1'-2 X 2'$
instead of $(0+1-2)X(0'+1'-2')$ under the condition (A') 
(see \refsec{sec:3compo}). Here $0,1,2$ means the three components of 
a eigenfunction as a 3-component function defined in
\refsec{sec:3compo}, $0',1',2'$ as well.

We have an error because of the separable form in the 3-component formalism.
Here we reorganize the discussion to evaluate the error
\cite{soler89,PAW} to fit to the formalism in this paper.
The error can be evaluated with a difference;
\begin{eqnarray}
(0+1-2)X(0'+1'-2')-(0X 0'+1 X 1'-2 X 2') 
= (0-2)X(1'-2')+(1-2)X(0'-2'), \nonumber \\
\label{eq:zeroonetwo}
\end{eqnarray}
%\end{widetext}
%We take these are 
%conttributions of a 3-component function within a MT site.
%If the right-hand side of \req{eq:zeroonetwo} is negligible,
%we can use the separable form $0 X 0'+1 X 1'-2 X 2'$,
%instead of $(0+1-2)X(0'+1'-2')$. 
Let us examine the error as the right-hand side of
\req{eq:zeroonetwo} under the assumption that $X$ is nearly spherical.
Remember that $(0-2)$ is completely zero if the condition (A) is
satisfied. When the condition (A') is satisfied instead, 
i.e., when we introduce the finite truncation parameters 
$l_{{\rm max},a}$ and $k_{{\rm max},a}$ (given after \req{f2}), 
we can expect that $(0-2)$ should contain high-energy remnant
(high angular momentum $l$ or highly oscillating remnant) with a small amplitude.
The remnant $(0-2)$ for each $L$ is largest at the MT boundaries.
In contrast, when (A') is satisfied, $(1'-2')$ is low energy part which  
converges quickly on the truncation parameters. 
The value and slope of $(1'-2')$ are zero at MT boundaries.
Thus we can expect the product $(0-2)(1'-2')$ should be small and
nearly orthogonal, i.e.,
$\delta n_a(\bfr)=(0-2)_a(1'-2')_a$ should satisfy $\int_a d^3 r\delta
n_a(\bfr)Y_L(\hat{\bfr}) \approx 0$ for low $L$.
Here suffix $a$ means quantities within MT at $\bfR_a$.
Based on these considerations we expect that the error affects little 
the total energy.
This can be checked by changing the truncation parameters within the PMT method.

This logic is applicable not only to the products of the eigenfunctions, 
but also to the electron density for the Coulomb interaction 
with some modifications.

%Then we expect that $(0-2)$ is becoming zero although $(1'-2')$,
%which contains only the low energy part changes little, when
%we enlarge the cutoffs.

%In practice, it is possible to use relatively small angular 
%momentum cutoff as $l_{{\rm max},a}\sim 4$ \cite{lmfchap}.

\begin{widetext}
\subsection{Atomic Force}
\label{sec:force}
全エネルギーの評価において、入力の電子密度$n_{\rm in}$に対して、
ポテンシャル$V=\frac{\partial (E_{\rm es} + E_{\rm xc})}{\partial n}$
を求め、それをハミルト二アンにもちいて、それの固有値問題を解き、固有値の和としてバンド
エネルギー$E_{\rm B}$を求めることができる。
また、その固有値問題の波動関数から電子密度$n_{\rm out}$を求めることができる。
収束すれば$n_{\rm in}=n_{\rm out}$が成り立っているが、収束途上ではなりた
たない。$n_{\rm in}\ne n_{\rm out}$のとき、
$n_{\rm in}$の汎関数としての全エネルギー(ハリスエネルギー)について考える。\\

First, we define the Harris energy $\ehf$ \cite{molforce,harris85} 
which is the total energy of a functional of the density; this gives
a reasonable estimate of the total energy even when the density is
somehow different from the converged density.
When not being converged yet, the input density $\nin$
must be treated as one generating the one-particle potential $V$,
and output density $\nout$ which is given from eigenfunctions
obtained from the eigenvalue problem of $V$. Here, $
V$ is given by \req{eq:v}. Now, $\ehf$ in the frozen core
approximation as a functional of $\nin$ is defined by \cite{molforce}:
\begin{eqnarray}
&&\ehf = E_{\rm k}^{\rm core} + E_{\rm B} - V[\nzc+\nin,\bfR_a] \cdot \nin 
+ E_{\rm es}[\nzc+\nin,\bfR_a] + E_{\rm xc}[\nzc+\nin],
\label{eq:ehf} \\
&&E_{\rm B} = \sum_p^{\rm occupied}
\alpha_{p}^{i*} 
\langle F_i|H^{\rm in}|F_j \rangle 
\alpha_p^j,
\label{eq:ebhf}
\end{eqnarray}
where $E_{\rm B}$ is the band energy.
$\alpha_i^p$ is the eigenvector of 
$\langle F_i|H^{\rm in}|F_j \rangle=
\langle F_i| \frac{-\Delta}{2m} + V[\nzc+\nin,\bfR_a] |F_j\rangle$.
Thus we have $E_{\rm B}=\sum^{\rm occupied}_p \epsilon_p$, where $\epsilon_p$ are eigenvalues.
The $\bfR_a$-dependence explicitly shown in \req{eq:ehf} is
through the $\MM$-transformation and $\calR$-mapping;
even when $\nzc+\nin$ is fixed as a 3-component function,
$\bfR_a$-dependence is introduced to \req{eq:v} through Eqs.(\ref{eq:barn0zcv},\ref{eq:calqdef}).
In addition, we have $\bfR_a$-dependence through $\nzc+\nin$.

Atomic forces are given as the change of the total energy 
for atomic displacement $\delta \bfR_a$.
Here let us consider the change of $\ehf$, written as $\delta \ehf$.
To obtain $\delta \ehf$, we use the derivative chain rule where we
treat $\ehf$ as a function of $\bfR_a$ through
$\{F_i(\bfr), \nin, \Vin, \bfR_a\}$; $\Vin$ means $V[\nzc+\nin,\bfR_a]$ in 
Eqs.(\ref{eq:ehf},\ref{eq:ebhf}).
Remember that there is $\bfR_a$ dependence through $\nzc$.
Here we assume the partial waves
($\{\phi_{al}(r),\dot{\phi}_{al}(r),\philo(r) \}$ in the case of the PMT
method) are not dependent on atomic positions as in Ref.\cite{molforce}.

Let us evaluate $\delta \ehf$. 
As for $E_{\rm B}=\sum_p^{\rm occupied} \epsilon_p$ 
as a functional of $\{F_i(\bfr),\Vin\}$, perturbation theory on
\req{eq:eigenp} gives
\begin{eqnarray}
&&\delta E_{\rm B} = \sum_p^{\rm occupied} \delta \epsilon_p
=\sum_p^{\rm occupied} \sum_i \sum_j 
\alpha_{p*}^i (\delta H_{ij} -\epsilon_p \delta O_{ij}) \alpha_p^j
=\delta \Vin\cdot \nout + \delta E_{\rm B}^{\rm Puley}, \label{eq:deleb}\\
&&\delta E_{\rm B}^{\rm Puley}=\sum_p^{\rm occupied} \sum_i \sum_j 
\alpha_{p}^{i*} (\delta H^F_{ij} -\epsilon_p \delta O^F_{ij}) \alpha_p^j,
\end{eqnarray}
where we have used $\delta (\Vin\cdot F^*_i F_j)=\delta \Vin 
\cdot F^*_i F_j + \Vin\cdot \delta (F^*_i F_j)$.
$\delta E_{\rm B}^{\rm Puley}$ is calculated from 
$\delta F_{0i}(\bfr)$ and $\delta C^i_{akL}$, which 
are given as a functional of $\delta \bfR_a$.

Since $E_{\rm es} + E_{\rm xc}$ is a functional of $\{\nin,\bfR_a\}$, we have
\begin{eqnarray}
&&\delta \ehf = \delta E_{\rm B} - \delta (\Vin \cdot \nin) 
  + \delta (E_{\rm es} + E_{\rm xc}) \nonumber \\
&&= \delta \Vin \cdot (\nout- \nin) 
  + \delta E_{\rm B}^{\rm Puley} 
  + \left.\frac{\partial (E_{\rm es} + E_{\rm xc})}{\partial \bfR_a}\right|_{\nin} \delta \bfR_a.
\label{eq:deltaehf} 
\end{eqnarray}
%Note that a contribution to $\bfR_a$-dependence is through $\nzc$.
There are three terms in the right hand side of \req{eq:deltaehf}.
The first term appears because $\ehf$ is not converged yet.

To calculate the first term, we need to know $\delta \nin$ which determines
$\delta \Vin$. When the self-consistency is attained and converged,
that is, $\nin=\nout$, $\delta \bfR_a$ uniquely determines
$\delta \nin=\delta \nout$.
However, this is not true when $\nin \ne \nout$.
In this case, there is no unique way to determine $\delta \nin$
for given $\delta \bfR_a$. Thus we need an extra assumption to
determine it. As a reasonable and convenient choice,
we use $\delta \nin=0$ in the sense of 3-component representation. 
Physically, this means that $n_{1,a}(\bfr)-n_{2,a}(\bfr)$  
together with frozen core centered at $\bfR_a$
moves rigidly to $\bfR_a+\delta\bfR_a$.
Then we can calculate corresponding $\delta \Vin$ through the change
$\delta \barnzcv_0$ in \req{eq:v}. $\delta \barnzcv_0$ is evaluated from
\req{eq:barn0zcv},
where note that $\nzc_0(\bfr)$ contains $\bfR_a$ dependence as
given in \req{eq:nzc}.\\


最後のパラグラフ:第一項(Forceの補正項):\\
Forceを計算するには、Harris-Folkner energy(Harris energyと同義)$E_{\rm HF}$の変分を
電子密度が完全にはself-consistentになってないとき、
とることを考えてやらねばならない。これは、上の式、すなわち、
\begin{eqnarray}
\delta E_{\rm HF} = (n_{\rm out}-n_{\rm in})\cdot \delta V
+\delta E_{\rm B}({\rm Pulay}) 
+\frac{\partial (E_{\rm es} + E_{\rm xc})}{\partial \bfR_a}
\end{eqnarray}
を計算することになる。ここで、第一項における$\delta V$としては、
$\bfR_a$を動かしたときに、どれだけのポテンシャル変化があるかを'適当に仮
定した電荷の変化'から求める。これは$n_{\rm out}-n_{\rm in}=0$で
あれば消える項であり、それなりに適切に推定できていれば十分である。
いくつかのオプションがありえるがまず単純なものは電荷の
第1,2成分のみがrigidにシフトするというものである。
これは、bndfp-dfrceによって評価されている。計算効率を
あげるため、電子系を完全には収束させずに原子位置の緩和をおこなう
必要があるが、その際、この補正は非常に重要である。


\subsection{onsite matrix}
\label{onsitematrix}
Here we summarize expressions of one-center matrix for $O_{ij},T_{ij}$, and
$V_{ij}$. These are essentially the same as what is shown in Ref.\cite{lmfchap}.
With the help of Eqs.(\ref{f2},\ref{f1}), Eqs(\ref{eq:norm},\ref{eq:kin},\ref{eq:v})
are reduced to be
\begin{eqnarray}
O_{ij} &=&\int_\Omega d^3r  F^*_{0i}(\bfr)F_{0j}(\bfr)
  + \sum_{akk'L} C^{*i}_{akL} \sigma_{akk'L} C^{j}_{akL}   \label{eq:normmat1}\\
T_{ij} &=&\frac{1}{2m_e}\int_\Omega d^3r  \nabla F^*_{0i}(\bfr) \nabla F_{0j}(\bfr)
  + \sum_{akk'L} C^{*i}_{akL} \tau_{akk'L} C^{j}_{ak'L},   \label{eq:kinmmat1}\\
V_{ij}&=&\int_\Omega d^3r  F^*_{0i}(\bfr)V_0(\bfr)F_{0j}(\bfr)
  + \sum_{akk'LL'} C^{*i}_{akL} \pi_{akk'LL'} C^{j}_{ak'L'} \label{eq:vpot},
where
\end{eqnarray}
\begin{eqnarray}
&&\sigma_{akk'l}= \inta d^3r  
 \left(\widetilde{P}_{akL}(\bfr) \widetilde{P}_{ak'L}(\bfr)
- {P}_{akL}(\bfr) {P}_{ak'L}(\bfr)\right), \label{matsig} \\
&&\tau_{akk'l}= \frac{1}{2m_e}\inta d^3r  
 \left(\nabla \widetilde{P}_{akL}(\bfr) \nabla \widetilde{P}_{ak'L}(\bfr)
-\nabla {P}_{akL}(\bfr) \nabla {Pq}_{ak'L}(\bfr)\right), \label{mattau}\\
&&\pi_{akk'LL'}= \sum_M Q_{kk'LL'M} {\cal Q}^{\rm v}_{aM} +
  \left(\barnzcv_{1,a} \circ \RR + v^{\rm xc}_{1,a} \right)\circ
  \widetilde{P}_{akL}(\bfr') \widetilde{P}_{ak'L}(\bfr')
- \left(\barnzcv_{2,a} \circ \RR + v^{\rm xc}_{2,a} \right) \nonumber \\
&&\circ
  {P}_{akL}(\bfr') {P}_{ak'L}(\bfr'), \label{matpi}\\
&&Q_{kk'LL'M}=\inta d^3r
\left( \widetilde{P}_{akL}(\bfr) \widetilde{P}_{ak'Ll}(\bfr)
- {P}_{akL}(\bfr) {P}_{ak'Ll}(\bfr)\right) \YY_M(\bfr) \label{qmom}. 
\end{eqnarray}
\end{widetext}
Note that $\sigma_{akk'l}$ and $\tau_{akk'l}$ are dependent only on
$l$ of $L=(l,m)$.
In Ref.\cite{lmfchap}, this $\pi_{akk'LL'}$ is further divided as
$\pi^{\rm mesh}_{akk'LL'} + \pi^{\rm local}_{akk'LL'}$.
${\cal Q}^{\rm v}_{aM}$ is given by \req{eq:calqdef}.

%%%%%%%%%%%%%%%%%%%%%%%%%%%%%%%%%%%%%%%%%%%%%%%%%%%%%%%%%%%%%%%%%%%%%%%%%%%%%%%%
\subsection{scalar relativistic approximation in the augmentation}
\label{app:srel}
Roughly speaking, it is allowed to take the scalar relativistic
(SR) approximation (e.g. see \cite{rmartinbook}) if we can safely replace the
non-relativistic (NR) wavefunctions with the SR wavefunctions within MTs.
The SR wavefunctions contain major and minor components. 
The major component should be smoothly connected to the NR wavefunction in 
the interstitial region, where the minority parts are negligible. 
All physical quantities within MT should be evaluated 
through the SR wavefunctions.
In the followings, we explain how the above idea can be implemented in the
3-component augmentation for bilinear products. 

First, we modify the 1st component of the basis. 
We use two component wavefunctions $\{{\mathtt g}_{1i,aL}(\bfr),{\mathtt f}_{1i,aL}(\bfr)\}$
instead of $F_{1i,a}(\bfr)$, where the SR approximation gives
${\mathtt f}_{1i,a}(\bfr)=\frac{1}{2m_e c} \frac{d {\mathtt g}_{1i,a}(\bfr)}{dr}$, 
where $c$ is the speed of light. 
For given $F_{0i}$ and $F_{2i}$ (they are the same as those of the NR case),
we ask the the major components ${\mathtt g}_{1i,a}(\bfr)$
to satisfy the boundary conditions as for value and slope at MT boundaries.
%Then we can determine $\{{\mathtt g}_{1i,aL}(\bfr),{\mathtt f}_{1i,aL}(\bfr)\}$ 
%instead of $F_{1i,a}(\bfr)$.

In order to calculate the contributions due to the 1st components within the SR approximation
instead of the NR approximation, we make a replacement 
$F^*_{1i,a}(\bfr)F_{1j,a}(\bfr') \rightarrow 
 {\mathtt g}^*_{1i,a}(\bfr){\mathtt g}_{1j,a}(\bfr')
+\left(\frac{1}{2m_e c}\right)^2 
 {\mathtt f}^*_{1i,a}(\bfr){\mathtt f}_{1j,a}(\bfr')$.
With this replacement, we can evaluate the density $n$, the matrix $O_{ij}$ 
and so on. This ends up with the total energy in the SR approximation.

Finally, we see that changes are in the replacement Eqs.(\ref{matsig}-\ref{qmom}),
where products $\widetilde{P}_{akL}(\bfr) \widetilde{P}_{ak'L}(\bfr)$ 
(and those with $\nabla$) should be interpreted not only from the products 
of the majority wavefunctions, but also from those of the minority.
This occurs also for the density $n_{1,a}$ included in \req{matpi}.

In such a way we can include the SR effect in the 3-component formalism.
In a similar manner, we can include the spin-orbit coupling in the 1st component, 
which results in the spin off-diagonal contributions \cite{chantis06a}.


%%%%%%%%%%%%%%%%%%%%%%%%%%%%%%%%%%%%%%%%%%
\subsection{choice of a real-space mesh}
\label{sec:realmesh}
(あたりまえな話かもしれない。実際には問題もありえる。読む必要なし)
For example, the smooth part of $E_{\rm xc}$ is evaluated on
the real-space mesh. Thus we wonder whether $E_{\rm xc}$ can be
dependent on choice of a real-space mesh even when we fix the total
number of mesh points. Here we show that there is no 
such dependence on a choice of the mesh.

First, let us specify our problem. We evaluate an integral 
$I=\int_\Omega f(n(\bfr)) d^3r$ as the sum on a discrete real-space mesh 
$\{\bfr_i\}$:
\begin{eqnarray}
I= \sum_{i} f(n(\bfr_{i})). \label{isum}
\end{eqnarray}
We now show $I$ is not dependent on the choice of the real-space mesh
under reasonable conditions.

The conditions are as follows:
\begin{itemize}
\item[(i)]
With finite number of $\{\bfG\}$, we represent $n(\bfr)$ for any $\bfr$ as
\begin{eqnarray}
n(\bfr)= \sum_{\bfG} n_\bfG e^{i \bfG \bfr}.
\end{eqnarray}
\item[(ii)]
Real space mesh $\{\bfr_{i}\}$ should correspond to the Fourier mesh of
$\{\bfG\}$ in the sense that
\begin{eqnarray}
\delta_{\bfG\bfG'}= \sum_{i} e^{i (\bfG-\bfG') \bfr_{i}},\nonumber \\
\delta_{ii'}= \sum_{\bfG} e^{i \bfG(\bfr_i-\bfr_{i'})}.
\end{eqnarray}
(For simplicity, we take $\sum_i$ and $\sum_\bfG$ include a normalization.)
\item[(iii)]
$f(n)$ is analytic on $n$ for positive $n$. This results in
	   an infinite series of expansion as 
\begin{eqnarray}
f(n)\!=\! f(\bar{n}) 
\!+\! f'(\bar{n}) (n-\bar{n}) \!+\! \frac{f''(\bar{n}) }{2!} (n-\bar{n})^2 \!+\!... , \label{fexpand}
\end{eqnarray}
where $f',f''...$ are derivatives with respect to $n$ at $\bar{n}$.
$\bar{n}$ can be average of density.
\end{itemize}

Our statement is that the integral $I$ is dependent on only $n_\bfG$, not
dependent on the choice of $\{\bfr_i\}$ as long as (i),(ii),(iii) are
satisfied. To verify this, it is enough to treat 
$I_m=\sum_i (n(\bfr_i))^m$, because
$I$ is given as a linear combination of $\{I_m\}$
due to (iii). As for $I_m$, we can easily show
\begin{eqnarray}
I_m = \sum_{\bfG_1+\bfG_2+...+\bfG_m=0} n_{\bfG_1} n_{\bfG_2} ... n_{\bfG_m}.
\end{eqnarray}
This means that the choice of real-space mesh do not affect $I_m$
as long as (i) and (ii) are satisfied. Thus our statement is verified.

This can be extended to the cases of GGA. 
This allows us to use relatively coarse mesh to calculate 
atomic forces because we have no artificial forces due to the choice 
of real-space mesh. However, we expect a numerical error when the real-space
mesh is too coarse since the expansion \req{fexpand} cannot be
converged rapidly.


%%%%%%%%%%%%%%%%%%%%%%%%%%%%%%%%%%%%%%%%%%%%%%%%%%%%%%%%%%%%%
%%%%%   QSGW                           %%%%%%%%%%%%%%%%%%%%%
%%%%%%%%%%%%%%%%%%%%%%%%%%%%%%%%%%%%%%%%%%%%%%%%%%%%%%%%%%%%%
%\widetext
\section{Implementatin of QSGW}
\label{sec:qsgwformalism}
QSGW法の具体的な実装について述べる。\\

\noindent 基本文献について:
%We use notations in 
基本的には文献\cite{kotani07a}でのノーテーションを用いる。ただ、現状のコードからみると記述が古くなってしまっている部分もある。
\begin{quote}
[文献8]T. Kotani, M. van Schilfgaarde, and S. Faleev, “Quasiparticle
self-consistent GW method: 
A basis for the independent-particle approximation,” Physical Review B, vol. 76, no. 16, Oct. 2007.
http://link.aps.org/doi/10.1103/PhysRevB.76.165106
\end{quote}

\noindent Acronyms:
\begin{enumerate}
\item[]
MPB: mixed product basis, which consists of PB and IPW.
\item[]
PB: product basis
\item[]
IPW: interstitial plane wave(=exp(iq r) but zero in MT). NOTE: we use
       two kinds of IPWs (for eigenfuncitons $|q+G|^2<$QGphi and for
       the Coulomb interaction $|q+G|^2<$ QGcou.)
\end{enumerate}

%%%%%%%%%%%%%%%%%%%%%%%%%%%%%%%%%%%%%%%
\noindent 課題:
PMT法の(LDA/GGAレベルでの)不安定性。要因は、1.コアとの直交性、2.基底関数が
null vectorを含むようになる、の2点にある。この不安定性をもうすこしきちん
ととりのぞきたい。1はMT半径をかなり小さくしなければphi,phidotとコアがお
よそ直行しているようでそれほどには問題にならないみたいだ。

PMT法では、3成分表示にのっとっているので、第0成分を「擬波動関数」とみな
せないことはない。ただ、現状のPMT法では,MTOについてはheadパートはそのまま用い
tailパートとAPWのみにプロジェクタを用いて、第0成分から第2成分をつ
くり、それに対応するphi,phidotを用いてaugmentしている;注意が必要。

\begin{enumerate}
\item
KMXAを小さくし(できたらPAW同様に2個にして)、安定化・効率化するには、よいプロジェ
クタが必要である。これには、各原子をpwemax=3程度で解いておいて、
そのときの、第2成分をプロジェクタとするのがよいと思える(2つ必要なのでそれと直行するエネルギー微分もとる)。
このプロジェクタを用いれば、今のようにhead partとtailパートを分別して取り扱う必要もなくせるかもしれない。

\item
計算の安定化:全エネルギーに「第2成分(=第0成分)」を固定するようなポテンシャルを加えて
計算を安定化させることも考えうる。
これは原子をPMT法でたとえばpwemax=3程度で解いたときの$a$サイトでの
第2成分$A_{2a}(\bfr)$(大きくなりすぎない妥当な第2成分の解。擬ポテンシャルの解にも相当する)
を用いて、$\Delta E= + \alpha \sum_p \sum_a \int_{a} d^3r (\psi_{2p,a}(\bfr) -A_{2a}(\bfr))^2$
をハミルトニアンに付け加えて解くことである。これによりいくらか
エネルギー的には損をすることになるかもしれないが、第2成分の不定
性が固定されることになる。(それゆえ第0成分も決定されてしまうことになる)。

\item 
QSGWを行うときには、MT半径を大きく取ったほうがPBを大きくとることが
でき効率的になるので有利(MT半径が小さいと、MPBにおいてIPWをかなり多く取
らないといけなくなり、相対的に計算が遅くなる)。
ただ将来的にはMT半径に依存しないような形でのMPBを利用したい。
PMT法自体はMTに重なりがあっても破綻はないアルゴリズムになっている。
GW法もそのようにしたい。
\item
bcc Fe単体でQSGW計算で正しくモーメントを出そうとすると(QSGWでもGGAとほと
んど同じモーメントが出るようだ)、3pをvalence扱いする必要あり。せめて
polarizationに入れなくてすむかどうか要チェック(現在調べ中)。
\end{enumerate}


%%%%%%%%%%%%%%%%%%%%%%%%%%%%%%%%%%%%%%%%
\subsection{波動関数の表現}
See eq.25 in the paper[TvSF].
\begin{eqnarray}
\Psikn(\bfr)
= \sum_{Ru}\alpha^{{\bfk}n}_{Ru} \varphi^{\bf k}_{Ru}({\bfr})
 + \sum_{\bf G}  \beta^{{\bfk}n}_{\bf G} P^{\bf k}_{\bf G}({\bf r}),
\label{def:psiexp}
\end{eqnarray}
である。前半がMT内パート、後半がMT外パート(IPW $P^{\bf k}_{\bf G}({\bf r})$で張られる)である。
MT内部は$\varphi^{\bf k}_{Ru}({\bfr})$という基底で張られている。これは$\bfk$をもつbloch和で、
\begin{eqnarray}
\varphi^{\bfk}_{R u}(\bfr)=
\sum_T \varphi_{Ru}(\bfr-\bfR-\bfT)\exp(-i \bfk(\bfR+\bfT))
\label{eq:blochsum}
\end{eqnarray}
である。ここで、$\varphi_{Ru}(\bfr)$は、MTサイトR内でのみnon zeroな関数
であり、uは、複合indexで「角運動量$L$,動径部分の区別の
index(phi,phidot,phizの区別)」である。
また、GW計算においてはコアは、MT内にあるものとして扱われる(ctrlでLFOCA=1にしておいてもLFOCA=0で再計算して用いられる)。
コア波動関数も上述のブロッホ和の形\req{eq:blochsum}で与えられている。

\noindent{\bf データファイル読み込みルーチン:}
\begin{itemize}
\item 
readeval:固有値読み込みルーチン
\item
readcphi: $\alpha^{{\bfk}n}_{Ru}$の読み込み
\item
readgeig: $\beta^{{\bfk}n}_{\bf G}$の読み込み
\item:
readngmx,readqg: ${\bf G}$の読み込み。
\end{itemize}
readevalなどを用いるには、init\_readeigen,init2\_reaeigenを先に呼ぶ必要
がある(これらはメインルーチンの中に書かないで一度目の呼び出しのときに。
すべてのファイルをランダムアクセスファイルにしたほうがいいような気もして
いる。メモリの制約で、少し大きいシステムになってくるとKeepEigen
off,KeepPPOVL offをGWinputでセットしないと走らない場合もある。
データへのアクセスは{\bf q}などを与えて直接に呼び出せるようにしている。
とにかく、あるプログラム(たとえばhx0fp0)を走らせるとき、それ以前に作ら
れ、ファイルに格納されたデータの(十分に高速な)読み出しが必要になる。
これを、read*という名前のルーチンで行えるよう統一したい。
書き出しに関しては、直接ファイルオープンして書き出させたほうが見やすい場
合もあるが、write*と言う名前のルーチンを対にしてモジュールに収めたほうがいいかもしれない。
また、データを二重に持たせているような部分は極力排除したい。


%%%%%%%%%%%%%%%%%%%%%%%%%%%%%%%%%%%%%%%%%%%%%%%%%%%%%%%%%%%%%%%%%%%%%%
\subsection{行列要素$\langle \psi |\psi M \rangle$の計算}
(以下の説明ではindexのシステムをもうすこし単純化、系統化したい)。

psicb,melpln,drvmelp2などの目標は、
最終的にzmelに格納される「行列要素$\langle\Psikqn(\bfr)|\Psi_{\bfk n'}(\bfr) M_{{\bf q}I}(\bfr)\rangle$」

を求めることである。MPBの$M_{{\bf q}I}$はPBとIPW(QGcou)からなる;
これらは$B_{\bfq Rm}(\bfr-\bfR)$と$P^{\bf k}_{\bf G}({\bf r})$など
と書かれる。(IPWはIPW(QGpsi)と二種類あるので区別しないといけない)。
plnとかmelpという文字の入ったルーチンはIPWに関わる部分である。
hsfp0*,hx0fp0*などにおいては、この行列要素を計算したのちに、
それを用いて、誘電関数やら、自己エネルギーを計算することになる。

以前、場合によっては、これらをすべてファイルにおとして
プログラムとして分割することも考えたが、メモリの関係で
オンザフライで求めている。drvmelp2がsxcfの方にのみ組み込まれているなど、
かなりごちゃごちゃした形になっている。すっきりさせたい。
高度な並列化を考える場合にはこの行列要素計算の並列化も必要になってくるか
と思う。計算量のオーダー的には、N$^3$だろう。\\

smbasis()は以前に開発した新たなMPBだがあまりご利益がないのでいまは使っていない。
これは、MT内とMT外がスムーズにつながるようなMPBのアイデア。
たぶん動かないし放置でいいが、記録として残してある。将来的には
『「MTO基底の積」+「APW(MT内部のaugmentの仕方は配慮する)』
をあらたなMPBにしたいと考えてはいる。


以下、個別のルーチンの説明。\\

\noindent {\bf psicb\_v3:} \\
(注意:コア波動関数とバレンス波動関数の直交性の問題があり、
ncore=ncc=0として分極関数にはコアからの寄与をいれないのを最近はデフォルトにして
いるのでhx0fp0*内では実質的にスキップすることになる)。\\
%We usually set ncore=0, that is, we essentially skips this routine in hx0fp0*.
%Thus we do not include polarization due to cores.)
このサブルーチンではmatarix element 
$\langle\Psikqn(\bfr)| \varphi^{\rm core}_{\bfk n'}(\bfr) B_{\bfq Rm}(\bfr-\bfR) \rangle$
および
$\langle\varphi^{\rm core}_{\bfk+\bfq n}(\bfr)| \Psi_{\bfk n'}(\bfr) B_{\bfq Rm}(\bfr-\bfR) \rangle$
%``core$\times$valence$\times$product basis''
を求める;これは最終的にzpsi2b(iblochq,ib1k,ib2qk)に格納される。
ここで、$\varphi^{\rm core}_{\bfk n}(\bfr)$はブロッホ和\req{eq:blochsum}の形で与えられた
コア波動関数を意味している。このとき$n$はバンドindexとみてよい
(コアについては、$n$は「原子サイト$R$、角運動量$L$、主量子数」からなる複合indexである)。

iblochは1:nblochで、nblochは、すべての原子についてのPBの個数の和である。
なので、nblochは、$B_{\bfq Rm}$における添字$Rm$の要素数でもある。
$R$はユニットセルの原子数、$m$のサイズは原子によって違っている
「各原子についてのPBの数」は”grep nbloch lbas”で表示される。

ib1kとib2kはバンドindexで、ib1kが1:nctotの間,ib2kが1:nccの間にあれば、そ
れらはコア波動関数である。またib1k,ibk2がぞれぞれnctot,nccよりも大きけれ
ば、それらはバレンス波動関数のindexとなっている。(なので、これらのindex
についてはコア+valenceで、通しでnでindexづけされている)。コアとしてどれだ
けのものをとるかはGWinputで指定できるがコアなしnctot=ncc=0がデフォルト。
(そもそもはnccはゼロで、ib1kが占有バンド、1b2qkは非占有バンドを表してい
たが、time-reversalを破る場合に都合が悪くなり、nccもあとから付け足した経緯がある)。

用いている式は、
\begin{eqnarray}
\langle\Psikqn(\bfr)| \varphi^{\rm core}_{\bfk n}(\bfr-\bfR) B_{\bfq Rm}(\bfr-\bfR) \rangle
= \sum_{u} \alpha^{\bfk+\bfq n}_{Ru} 
\langle \varphi^{\bf k+q}_{Ru}(\bfr)|\varphi^{\rm core}_{{\bf k} n}(\bfr) B_{\bfq Rm}(\bfr) \rangle
= \sum_{u} \alpha^{\bfk n}_{Ru} 
\langle \varphi_{Ru}(\bfr)|\varphi^{\rm core}_{n}(\bfr) B_{Rm}(\bfr) \rangle
\nonumber \\
\end{eqnarray}
である. $\langle\varphi^{\rm core}_{\bfk+\bfq n}(\bfr)| \Psi_{\bfk n'}(\bfr) B_{\bfq Rm}(\bfr-\bfR) \rangle$
についても同様である(nccがゼロでない場合)。

GWコードではiclass(i)=iatomp(i)となっており、多くのことろでiclassはダミー
である。ただ、CLASSファイルに等価原子位置の情報が格納されており、これを
読み込んで波動関数を空間群で回転させるときなどに使っている。

 \\

\noindent {\bf psi2b\_v3:} \\
このサブルーチンではmatarix element 
$\langle\Psikqn(\bfr)| \Psi_{\bfk n'}(\bfr) B_{\bfq Rm}(\bfr-\bfR) \rangle$
を求めている。valenceの$\Psi$に関わる部分のみである。用いている式は、
\begin{eqnarray}
\langle\Psikqn(\bfr)| \Psi_{\bfk n'}(\bfr) B_{\bfq Rm}(\bfr-\bfR) \rangle
= \sum_{u} \sum_{u'} \alpha^{\bfk+\bfq n}_{Ru} \alpha^{* \bfk n'}_{Ru'}
\langle \varphi^{\bf k+q}_{Ru}(\bfr)|\varphi^{\bf k}_{Ru'}(\bfr) B_{\bfq
Rm}(\bfr) \rangle \nonumber \\
=
\sum_{u} \sum_{u'} \alpha^{\bfk+\bfq n}_{Ru} \alpha^{* \bfk n'}_{Ru'}
\langle \varphi_{Ru}(\bfr)|\varphi_{Ru'}(\bfr) B_{Rm}(\bfr) \rangle
\end{eqnarray}
である。変数cphikには$\alpha^{\bfk n'}_{Ru'}$が格納されている。cphikqについても同様である。
ppbに$\langle \varphi_{Ru}(\bfr)|\varphi_{Ru'}(\bfr)
B_{Rm}(\bfr)\rangle$が格納されている;これは実関数である(実球面調和関数、実数のradial関数を使っているので)。
ppb(nc+1,...)などとなっているのはncがコアの数でその分オフセットする必要があるからである。\\


\noindent {\bf melpln2:}\\
名前はmatrix element for plane waveを略したつもりで、melplnになった。
このルーチンでは、
$\langle\Psikqn(\bfr)| \Psi_{\bfk n'}(\bfr) P^{\bf q}_{\bf G}({\bf r}) \rangle$
を求めている。なので、これと、psi2b\_v3,psicb\_v3などの結果とあわせて、すべてのzmelを求めたことになる。
使っている式は、
\begin{eqnarray}
\langle\Psikqn(\bfr)| \Psi_{\bfk n'}(\bfr) P^{\bf q}_{\bf G}(\bfr) \rangle
= \sum_{G'} \sum_{G''} \beta^{\bfk+\bfq n}_{\bfG} \beta^{* \bfk n'}_{\bfG'}
\langle P^{\bf k+q}_{\bf G'}(\bfr)| P^{\bf k}_{\bf G''}(\bfr) 
P^{\bf q}_{\bf G}({\bf r}) \rangle \label{eq:ppip} 
\end{eqnarray}
であるが、この最後の部分は
\begin{eqnarray}
\langle P^{\bf k+q}_{\bf G'}(\bfr)| P^{\bf k}_{\bf G''}(\bfr) 
P^{\bf q}_{\bf G}({\bf r})\rangle
=\langle \exp(i (\bfG'- \bfG'')\bfr) |\exp(i \bfG\bfr)\rangle_{\rm I}
\label{eq:ppx}
\end{eqnarray}
と書ける。ここで最後の$\langle ... \rangle_{\rm I}$はinterstitialのみでの積分を表す。
この行列要素は、その足が$(\bfG'- \bfG'', \bfG)$となっている。
$\bfG'- \bfG''$の取りうる範囲は、波動関数の積であり、その絶対値は2倍の大きさまでとり
得ることになる。この\req{eq:ppx}の行列要素が、ppx(1:ngc2,1:ngc)である。ngcは、
QGcouに対応するIPWの個数である。ngc2は、$\bfG'- \bfG''$の取りうる範囲の
個数であり、およそ2$^3$倍になっている。
getppxをコールすると、ppxに値が格納される仕組みになっている
(モジュールを用いている。便利だけど分かりにくい)。

計算手順としては、まず、
$\beta^{\bfk+\bfq n}_{\bfG} \beta^{* \bfk n'}_{\bfG'}$
の積を作ってgg(n,n',$\bfG-\bfG'$)に格納している。
そのあと最後に、call matm(ppx,gg,zmelp,ngc,ngc2,ntp0*nt0)
を行って、この\req{eq:ppip} の左辺を得ている。

このルーチンの面倒な点は、エネルギーカットオフをしているので
$\bfq$点ごとに、$\bfG$のとりうる範囲がちがうことである。
それで、PPOVLファイル内のppxも$\bfq$ごとに作り直しており、
ファイルがそれで巨大になってしまっている(PPOVLはrdata4gw.m.Fで作っている)。

ただこれは、ばかげたやりかたです; 範囲が違うだけであって、ppxすなわち
$\langle \exp(i (\bfG'- \bfG'')\bfr) |\exp(i \bfG\bfr)\rangle_{\rm I}$
の値自体は、$\bfG'- \bfG''-\bfG$のみに依存しているだけで,
そんなに大きいことはなく、$|\bfG(\Psi)+|\bfG(\Psi)|+|\bfG({\rm Coulomnb})|$よりも小さなGに対して、
積分を計算しておけばいいはずである。このあたり作り方がへたすぎる。
それから、すぐにggxは作れるはずである。

shtvはhx0fpのほうではゼロにセットされているようである。\\

\noindent {\bf drvmelp2:}\\
このルーチンは、melpl2を被って便利になるように作った。ドライバーというつもり。
本来はたぶん、hx0fp0\_scでも使うつもりだったとおもうが中途半端なことになって
いる。あまり多重にサブルーチン化してもご利益はなく、やめたほうがいいのかもしれない。


\section{Not organized: memorandum}

\begin{enumerate}
\item
計算機においては有限の変数(離散化)により全エネルギーを表現することが
必要である。そして、そのエネルギーを最小化することで全エネルギーを決定することができ
る。すなわち、Kohn-Sham方程式を出してから離散化するのではなく、
全エネルギーを離散化表示してから、離散化されたKohn-Sham方程式を導出する
のである。これにより、全エネルギーとそれの微分であるハミルトニアンとの整
合性がきちんと保たれることになる(この場合、打ち切り誤差がなければ厳密にその関係性が満たされる)。
このことを全エネルギーの標識において確認し、どれだけの離散化
パラメーターが出てくるのかを概観する.このとき、radial方向の積分に関して
は離散化誤差は無視する;この離散化は、計算速度にあまり影響せず
十分に細かくとれるからである。まず、周期境界条件により、波動関数は1stBZ中の波数$\bfk$をもつことになる。
1stBZを格子で切り分け格子点の上でのみの$\bfk$点を考えるようにする。
それで、envelope関数$F_{0i}$のindex $i$は、この離散化された$\bfk$の
値を含む複合indexとなる。

envelope関数$\{F_{0i}\}$は、「平面波(波数$\bfk+\bfG$で指定される)」
と「smooth Hankel関数をBloch sumして$\bfk$を持つようにしたもの
(さらには、原子位置、角運動量、およびsmooth Hankelのdamping因子EHと
smoothing radius RSMHで指定される。)」を含んでいる。
これらをaugmentするときの係数$C^i_{akL}$と$P_{akL}$は、
augmentationの処方箋を決めれば決定される。現在、この処方箋は、
\cite{Bott98}のXIIに書かれている手法で行われている。
augmentationに関しては、$k$のカットオフKMXA(KMXAはゼロからなのでKMXA+
1個),$L$のカットオフLMXAが必要となる。$\tilde{P}_{akL}$は、$\phi_l(r),\phidot_l(r)$という二つの
radial関数をもちいて組み立てられる。これらは、onsiteでの
$\tV_{1a}$(の球対称部分)を用いてradial schrodinge eq.を適当なエネルギー
$\epsilon_{al}$で解くことで得られる。この「$\phi_l(r),\phidot_l(r)$」
を固定する(オプションFRZWF=1)ことも可能である
(ただ、計算を通じて固定するには、それを解くエネルギー$\epsilon_{al}$も
固定する必要がある。通常は、$\epsilon_{al}$には占有バンドの重心位置、あるいは、fermiエネル
ギー(セミコアがあるとき)が選ばれる。SPEC\_IDMOD=1でこのpnuが固定でき、FRZWVで
ポテンシャルも固定できてるはずだが、これで本当に基底の固定が完全にできているかは要チェック)。
将来的にはPAW法と同様に「基底を完全に固定」したほうがいいのではないかと思う。
frozenフォノン計算などには有利である。

それで、この方法は、全エネルギー最小化に関しておおむねの最適化
になっていると言えるが、上述のように、「基底の固定(phi,phidotの固定)」
が通常のオプションではなされないことには注意する必要がある。また、格子変
形したときなど、Gベクトルの数が急に変化してエネルギーが突然変化する、
という点もありえる(ctrlにおいて、GMAXの変わりにFTMESHを用いればさけれる)。

\item
上述の基底関数をもちいて、運動エネルギーの行列要素を計算す
ることができる。この式の第0成分はenvelop関数にのみ関係しており、その運動エネルギー行
列要素は解析的に計算できる。\\
コードでは,bndfp-hambl-smhsbl-hhibl,hhibl内。smooth Hankelの積やそれ
をLaplacianではさんだものの積分がsmooth Hankelで書けること---\cite{Bott98}の
(10.10),(10.12)など---を利用してこれを計算している.
第1,第2成分に関しても上述の$C^i_{jkL},{P}_{akL},\tilde{P}_{akL}$から計算
できる。

\item
$E_{\rm es}+E_{\rm xc}$を評価することを考える。
まず、第1,2成分については、radialな積分に帰着できるので、運動エネルギー
の時と同様に$C^i_{jkL},{P}_{akL},\tilde{P}_{akL}$から計算できる。
次に、第0成分について考える。
このためには、envelope関数(smooth Hankelと平面波)を、
実空間メッシュ(実空間でのユニットセルを等間隔で分割したもの)の上で表現
したものを考える必要がある(コード内では,bndfp-addrbl-rsibl-rsibl1,rsiblpで、
固有関数の実空間メッシュのうえでの値を直接に生成している)。
そして、それの積により電子密度smrhoをrsibl2で作っている(このsmrhoが
bndfp.Fに返されmkpotに渡されてポテンシャルの生成に使われる);
電子密度は実空間メッシュの上での値で与えられるわけである。
これをFFTすると電子密度は$\bfG$で展開して表現されていることになる。
$E_{\rm es}$の評価には、この表現を用いる。coreからの寄与や多重極変換にか
らんで、$\tnT$には、smooth HankelやGaussianを含むがこれらに
関係する部分は解析的に取り扱う(mkpot-smvesのコメントは、参考になる。しか
し、本書の方が正確である---coreのsmooth Hankelによる寄与が適当に省略されて書かれている)。
また、$E_{\rm xc}$の第0成分の評価においては、実空間メッシュの上で交換相関エネ
ルギーを求めてそれを積分することをおこなう\\
(mkpot-smvxcm. mkpot-smvxc2はvalence
のみに関する量を計算している.mkpot-smvxc2はおそらくvalenceのみの寄与を表示するこ
とのみに必要で、sc計算には不要)。
\end{enumerate}

以上で、全エネルギー$E$が計算機においてどのように表現されているかを与えられる。
subroutine mkpotは与えられた$n$からこれらのpotentialを計算するルーチンである。

% これらを考え合わせると、けっきょく、全エネルギーが、$\rho_{ij}$の汎関数で与
% えられることになる。$\rho_{ij}$は、この全エネルギーを最小にするように決定される。
% それで、これを調整してエネルギーを最小化する問題になる。
% {\small 注意:金属の場合は、$\rho_{ij}$の汎関数というよりも,それを生成す
% る$H_{ij}$(あるいはポテンシャルの)汎関数と考えるほうがよい。
% まず$H_{ij}$が与えられれば、BZ内の積分も考慮した形で$\rho_{ij}$が与えら
% れる(離散的な$\bfk$点でtetrahedron法をもちいる)。
% これを用いて全エネルギーが計算される。
% 一般的に言っても、密度の汎関数というよりそれを生成するポテンシャルの汎関数と
% して考察したほうが便利な場合も多い。}

%\subsection{全エネルギーの最小化}
% これで、\req{eq:etot0}における全エネルギーをあたえたことになる。
% これの最小化問題を解くには,$E_{\rm es} + E_{\rm xc}$の
% $n$による汎関数微分を計算し、$V_{ij}$を得る必要がある。
% この際、$n$は3成分からなるので、そのおのおのの成分について微分をとり、
% $\frac{\delta E_{\rm es}}{\delta n_0} \frac{\delta n_0}{\delta\rho_{ij}}$
% などの和として,$V_{ij}$は計算される。ここで、
% $\frac{\delta n_0}{\delta \rho_{ij}}$は、\req{eq:denmat}により容易に、
% 第0成分からの寄与$F_{0i}^*(\bfr) F_{0j}(\bfr)$であることが
% 分かる;それで結局、$V_{\rm es0}(\bfr)=\frac{\delta E_{\rm es}}{\delta n_0(\bfr)}$
% を用いて,$\langle F_{0i}(\bfr)|V_{\rm es0}(\bfr)| F_{0j}(\bfr)\rangle$
% として計算できることになる。第1,2成分についても同様である。

% それで、$\frac{\delta E_{\rm es}}{\delta n_0}$を
% 3成分それぞれについての計算することが必要である。


%\begin{eqnarray}
%&&H_{ij} \equiv \langle F_i| \kin+ V |F_j \rangle 
%= \int dr F^*_{0i}(\bfr) (\kin+V_0(\bfr)) F_{0j}(\bfr)  
%+ \sum_a \sum_{kL} (C^i_{akL})^* \sum_{k'L'} C^{j}_{ak'L'} \\
%&&\times
%\left(
% \int_{|\bfr| <R_a} \hspace{-15pt} d\bfr \tilde{P}_{akL}(\bfr) \left(\kin + V_{1a}(\bfr) \right) \tilde{P}_{ak'L'}(\bfr)
%-\int_{|\bfr| <R_a} \hspace{-15pt} d\bfr       {P}_{akL}(\bfr) \left(\kin + V_{2a}(\bfr) \right)       {P}_{ak'L'}(\bfr) \nonumber
%\right).
%\end{eqnarray}

% ここで、\req{eq:ees2}の書き方をもちいれば、
% \cite{lmfchap}のEq.(26)における$\tV_0$などは、
% $\tV^{\rm es}_0= v\cdot \tn_0$,$\tV^{\rm es}_1= {\cal R} \cdot\tn_1$,
% $\tV^{\rm es}_2= {\cal R} \cdot \tn_2$と書くことができる。
% $n$での汎関数微分をとる際には、$E_{\rm es}$は直接的には、
% $\tn$の関数であることに注意をする必要がある。$\tn$は$n$の汎関
% 数でありその関係は\req{eq:val0}-\req{eq:val2}で示されている。
% それで、$\tn$は、${\cal G}(n_1)$などを含んでいるから、
% それらを$n_1(\bfr)$で汎関数微分すると、\req{eq:gdef}の定義にしたがい、
% \req{eq:qal}を通じて、$r^l Y_L(\hat{\bfr})$に比例する項が出てくることになる。
% [$\tV^{\rm es}$は、$E_{\rm es} + E_{\rm xc}$の
% $n$による汎関数微分において、${\cal G}$を通じての微分を含まない部分として定義できる]。

% それで結局、最終的に得られる静電エネルギーの$V_{ij}$への寄与は、
% \cite{lmfchap}のEqs.(26)-(29)で与えられる。このEqs.(26)-(29)では、$V$は
% 静電相互作用のみのものであると考え、$\tV_0$は、本書の$\tV^{\rm es}_0$
% を意味すると読めばよい。チルダのついているのは多重極変換後の量であり、
% $V^{\rm es}_1$にはチルダがついてないが、これは、多重極
% 変換で、$n_1=\tn_1$であるからである(それで\cite{lmfchap}はすべての第1
% 成分にはチルダをつけてない)。

% また、交換相関エネルギーの$V_{ij}$への寄与は、単純に\req{eq:exc}の
% $n_0,n_1,n_2$を介しての汎関数微分で与えられることになる。
% それらを$V^{\rm xc}$とおくと,次式を得る。
% \begin{eqnarray}
% &&V^{\rm xc}_{ij} 
% = \int dr F^*_{0i}(\bfr) V^{\rm xc}_0(\bfr) F_{0j}(\bfr)  \nonumber \\
% &&+ \sum_a \sum_{L,L'} (C^{i}_{akL})^* \sum_{k'} C^{j}_{ak'L'}
% \left(\int_{|\bfr| <R_a} d\bfr \tilde{P}_{akL}(\bfr)V^{\rm xc}_1(\bfr)
% \tilde{P}_{ak'L'}(\bfr)
% -\int_{|\bfr| <R_a} d\bfr {P}_{akL}(\bfr) V^{\rm xc}_2(\bfr) {P}_{ak'L'}(\bfr) \right)
% \label{eq:vxc}
% \end{eqnarray}
% これは、\req{eq:over}において、$V^{\rm xc}$をはさんだ形である。

% $\tV=\tV^{\rm es}+V^{\rm xc}$は、$E_{\rm es} + E_{\rm xc}$の
% $n$による汎関数微分において、${\cal G}$に関わらない部分として定義できる。
% これは、\cite{lmfchap}では、Eqs(41)-(43)における定義と一致している。

% \noindent [注意: 
\cite{lmfchap}は$V^{\rm xc}$の扱いをすこし間違っている(lmfのコードは
まちがってない)。$\tV$から、そのなかのEqs.(26)-(29)で、$V_{ij}$が
作れるように書いてある。しかし、これだと、$V^{\rm xc}$に
関してまでも、多重極変換が関係してるような表式になってしまい、間違ってい
る。ただしコードは間違っていない。

mkpot-locpot-augmat-gaugmにおいては、
gpot0,gpotb(これがポテンシャル$\times r^l Y(\hat{\bfr})$の積分)が
使われているがこれらは静電ポテンシャルのみから計算されている。]


% xxxxxxxxxxxxxxxxxxxxxxxxxxxxx

% これが数学的にどう組み立てられているかに注意する。
% 単純には、基底関数と、ポテンシャル$V$の汎関数となっている。
% この$E$は3成分表示での波動関数や密度、ポテンシャルの表現を用いて定義され
% ており、この$E$を与えた時点で、問題は既に簡略化されていて数値的に解ける問題と
% して意味を持つことに注意する。

% まず、のちほど述べるように第一項$(E_{\rm B}^{\rm core}-V \cdot n^{\rm core})$ 
% はコアの運動エネルギーであり、$E$を最小化にするという
% 点からみれば、定数としてよい。

% 二項めのバンドエネルギー$E_{\rm B}$は$V$の関数であり、
% $E_{\rm es}[\tnT]$,$E_{\rm xc}[n^{\rm c}+n]$であったが、$\tnT$は
% $\nT$の関数であるとみなせることに注意する。また、\req{eq:eig}から、$n$を
% (基底関数を固定しておけば)$V$の関数であると考えることができ、コアの電
% 子密度は以下で述べるように固定されていると考えると、
% 全エネルギーは$V$の汎関数$E[V]$であると考えることができる。
% %コアからの外場は、$E_{\rm es}$に含まれている.

% もっと正確にいえば、基底関数をaugmentするためのphi,phidotは
% $V$(正確には$V_1$)によって与えられるから,そこまで含めて、
% この$E$を$V$の関数と考えることができる。ただしphi,phidotは、局所的に
% およその最適化をはかったに過ぎず、以下でエネルギー最小化の方程式を出すと
% きに、基底関数に関して変分をとっているわけではない。

% $V$の各成分は独立であると考えてよいことに注意する(最小化をはかるときに
% 重要である)。すなわち、$V_0,V_1,V_2$を独立に動かして、
% 最小エネルギーを計算することになる.


% 原子核位置$R_a$依存性がどこに入っているかに注意しておく。
% これは、$n^{\rm c}_a(\bfr-\bfR_a)$の3成分表示の形で、
% $E_{\rm es}[\tnT]$,$E_{\rm xc}[n^{\rm c}+n]$の$\tnT,n^{\rm c}$
% に入っている(第0成分にのみ$\bfR_a$依存性は入ってくる)。また、
% augmentationの操作において、$C^{i}_{akL}$にも入っている。
% これらのことは以下でForceを計算する時に問題になる。

% ているから、$\{ R_a\}$と3成分表示の$\nT$の関数として、次章で定義される。
% また、$E_{\rm B}$は、$V$と
% $\{F_i\},P_{akL},\tilde{P}_{akL},C_{akL}$の関数であった;これらは
% $R_a$依存性を持ちうる。たとえば、原子を動かしたとき、$\{F_i\}$を変化させ
% ない場合でも、$C_{akL}$などは変化する。
% 原子核を動かしたときに全エネルギーがどう変化するかはForceのところで考える。
% とにかく、3成分表示にもとづいて考える必要がある;たとえば「3成分表示にお
% いて原子核をうごかしたとき$n$を固定しておく」というのは、もともとの対応する電子密度で言えば、
% すこし変な電子密度の動かし方をしてることになる;このときには、$n_2(\bfr)$成分が
% $n_0(\bfr)$成分をくりぬいたものではなくなる。が3成分表示を基礎におけば問題は生じない。
% \req{eq:deltae}あたりで説明する。).


% \subsection{Harris-Folkner energyとKohn-Sham energy}
% \subsubsection{KS energy}
% まず、通常のKohn-Sham energyは密度の汎関数としての全エネル
% ギーであり, 密度$n_{\rm in}$に対して$n_{\rm out}$をもとめたのち、
% この$n_{\rm out}$に対するKS energyが、
% \begin{eqnarray}
% E_{\rm KS} = E_{\rm k}^{\rm core} + E_{\rm B} - V \cdot n_{\rm out} + 
% E_{\rm es}[n_{\rm out}] + E_{\rm xc}[n_{\rm out}]
% \label{eq:etotks}
% \end{eqnarray}
% として計算することができる(ここではcoreの密度は略す)。
% ここで、$V$は$E_{\rm B}$計算時に使われる一体ポテンシャルであり、
% $V=\frac{\partial (E_{\rm es} + E_{\rm xc})}{\partial n}$
% を$n_{\rm in}$で計算するものである。このとき、
% $E_{\rm B} - V \cdot n_{\rm out}$は、$n_{\rm out}$に対する運動エネルギー
% となっている。この$E_{\rm KS}$の変分は、
% \begin{eqnarray}
% \delta E_{\rm KS} = (-V-V[n_{\rm out}])\cdot \delta n_{\rm out}
% \label{eq:detotks}
% \end{eqnarray}
% と表される(少し計算するとわかる)。ここで、$V[n_{\rm out}]$は、$n_{\rm
% out}$で計算したポテンシャルである。\\


% \noindent * bndfp-mkekinで$E_k$の計算を行っている。

% \subsubsection{Harris Folkner energy}
% Harris Folkner energyは
% \begin{eqnarray}
% E_{\rm HF} = E_{\rm k}^{\rm core} + E_{\rm B} - V \cdot n_{\rm in} + 
% E_{\rm es}[n_{\rm in}] + E_{\rm xc}[n_{\rm in}]
% \label{eq:etothf}
% \end{eqnarray}
% として定義される。$V$はKS energyとおなじく$n_{\rm in}$に対して計算される
% ものである。KS energyと比較すると、$E_{\rm B}$は同じであるが,それ以外の
% 項は$n_{\rm in}$を用いている点で違っている。この$E_{\rm HF}$の変分をとると、
% \begin{eqnarray}
% \delta E_{\rm HF} = (n_{\rm out}-n_{\rm in})\cdot \delta V
% \label{eq:detothk}
% \end{eqnarray}
% となることがわかる。
% lmfでは、以下でのForce計算にはおもには、$E_{\rm HF}$をもちいる。
% そこでは、収束が甘いときのForceの補正項を計算するのに\req{eq:detothk}を用
% いている。


% \subsection{Forceの計算}
% lmfにおいて、原子位置をずらしたときの\req{eq:etot0}の変化を考える。
% このエネルギーは密度行列$\rho(\bfr,\bfr')$と原子位置$\bfR_a$の関数であると
% 考えられることに注意する。核からの電荷$n^{\rm Z}$、コアの電荷$n^{\rm c}$
% は原子位置とともに動く。$E_{\rm es}$は\req{eq:ees2}で与えられており、
% $\tnT$は\req{eq:all0},\req{eq:all1},\req{eq:all2}であたえられていた。
% これらを考えると、原子位置$\bfR_a$は直接には、第0成分\req{eq:all0}を通じ
% てのみ$\bfR_a$に依存している(3成分表示で考えるときには、もとの空間で考
% えるのと意味が違っていることに注意する)。また、$E_{\rm xc}$についても同
% 様である。

% それで、\req{eq:etot0}の変分を考えると、$\bfR_a$を固定していた時には、
% $\delta E = \frac{\delta E}{\delta \rho(\bfr,\bfr')}\delta
% \rho(\bfr,\bfr')$と書けることを考えると、
% \begin{eqnarray}
% \delta E = H_{ij} \delta \rho_{ij} + \overline{\delta H}_{ij} \rho_{ij}
% +\frac{\partial (E_{\rm es} + E_{\rm xc})}{\partial \bfR_a}
% \label{eq:detot}
% \end{eqnarray}
% と書くことができる。self-consistentになっていれば第一項はゼロである。
% 第二項の$\overline{\delta H}_{ij}$は基底$\{F_i\}$が変化す
% るための変分を意味する。この第二項がPulay force $\delta E_{\rm B}({\rm Pulay})$である。
% 第3項がエネルギー変化のメインの項であるが、上述のように第0成分を通じての
% みの変分を考えればよい。

% それで、$\delta E_{\rm B}({\rm Pulay}) = \overline{\delta H}_{ij} \rho_{ij}$は、
% \begin{eqnarray}
% \delta E_{\rm B}({\rm Pulay}) = \sum_p^{\rm occupied}
%  (\delta\epsilon_p)^{\rm Pulay}
% =\sum_p^{\rm occupied} \sum_i \sum_j 
% (\alpha^p_i)^* (\delta^R H_{ij} -\epsilon_p \delta^RO_{ij}) \alpha^p_j.
% \end{eqnarray}
% と書くことができる。
% ここで、 原子位置の変化に対して定まる
% $\delta^R H_{ij}$ と $\delta^R O_{ij}$ は、
% $F_{0i},C^{i}_{akL}$を介しての変化量である。
% [あえていうなら $\tilde{P}_{akL}(\bfr)$,${P}_{akL}(\bfr)$の変化を介して
% の変化量もあるが、それは無視する。そもそもこれらの量はエネルギー最小でき
% められたわけでもない。]


\subsection{結晶格子の変形}
(まだimplementされていない)。
結晶格子を変形させる。格子が変形すると基底関数がずれる。
これにより電子密度がどう変化するか?などを考えればよい。
このとき係数$\rho_{ij}$は固定されている。
[原子が平衡位置にあるなら、それを介しての微分はゼロであり、
格子を変形するにつれて原子を動かしても動かさなくてもエネルギー変化には寄与しない。]
また、第0成分のみ考えればよい。
そもそも、応力を求めるときの計算量は、歪みテンソルの数が6個であり、
最大でも(真面目に数値差分で計算しても7倍の計算量ですむ)。
それにある意味、完全に並列化ができる。実装する値打ちがあるかは検討課題。。



\subsection{電荷の表現}
\begin{itemize}
\item 
bndfpがlmfpから呼ばれる。バンド計算のコア部分.ここで電子状態に関する一回
のiterationをおこなう。iterationのループはlmfpがうけもつ。
[LDA+Uに関しては、d-channelの占有数matrixをlmfpの部分で扱っており、
混乱ぎみの構造になっている。]
c-preprocessorでlmfgw(GW計算のドライバー)も埋め込まれているのでややこし
い。mkpotで与えられた電子密度からポテンシャルを作っている。
bndfpでは$n_{\rm in}$と$n_{\rm out}$を区別して扱っている。
電子密度$n_{\rm out}$は、hamblでバンド計算した後、bndfp-addrblで生成される。
以下のほとんどがbndfpから呼ばれているルーチンの説明である。

\item 
$n_{\rm in}$について。

これのvalenceの部分は、\req{eq:nval}で定義された$n$の表示で与えられている。
bndfp内において、
smrho(k1,k2,k2,nsp)が第0成分。localなdensityはw(orhoat(ix,ibas))に格納し
てある(説明は以下)。コード内ではorhoat=w(oorhoat)なので注意がいる;ポインタ
のポインタの構造。これらがmkpotに渡される。このw(orhoat(ix,ibas))の中身は、
rhoat(nr,nlml,nsp) = w(orhat(ix,ibas))であり、YLM展開の電子密度を
含む。nrがradial mesh,nlmlがL、nspはスピン。ixは1,2,3をとる。
ix=1,2が$n$の第1,2成分に対応している。
第3成分には、$ n^{\rm core}_{\rm true\_onsite}(\bfr)$
が格納されている。

これらに\req{eq:nc}の$n^{\rm c}$を加えれば、全電子密度が得られる。
たとえば、mkpotには、smrho,rhoatが渡され、これらがmkpot内部で付加されて、
全電子密度$n+n^{\rm c}$を作り、それを用いて、交換相関ポテンシャルを生成
する仕組みになっている。このとき、\req{eq:nc}をみればsmooth Hankelの
tailの寄与を付加しなければならないことがわかる。これは下のほうで説明する。

\item
$n_{\rm out}$について。

w(osrout)がsmrho,w(orhoat1(:,:))がw(orhoat(:,:))
に対応する。bndfp-mixrhoが$n_{\rm in}$と$n_{\rm out}$のmixing routine
である[orhat1とw(oorhat)が並べてあり整理されてない].

\item
電子密度は、iodenにおいて書き出されている。
これは、lmfpにおいて、bndfpをcallしたあとに実行される。bndfpの結果の電子密度
がw(osmrho),w(oorhat)を介してわたされる。{\small[このあたりのデータフローの書き
     方もかなり汚い。osmrhoは構造体potに含まれるポインタでこれがupackさ
     れることで、lmfp,bndfpのどちらにおいてもアクセスできることになる。
     Markさんとしたらpotからいれたり出したりしてるわけだが、構造体の乱用
     と思える。構造体ならpot\%smrhoなどになるので少し明瞭。しかし
     fortran2003にならないとallocatable arrayを構造体にいれれない。
     そもそもonce-writeでないものを構造体にいれるのはさけるべき。]}


\item
電子密度$n^{\rm c}+n$を書き出すには、iodenをつかわなくてもいいかもしれない。
確実なやり方は、$V_{\rm xc}$が$n^{\rm c}+n$から作られているところを
調べることである。この部分を改良して電子密度を書き出すのがよいかもしれな
い。開発者も理解しやすいので
[ここを理解しておくと、$V_{\rm xc}$を書き換えるにも都合がよい]。
$V_{\rm xc}$の第0成分は、mkpot-smvxcmで、
$V_{\rm xc}$の第1,2成分は、mkpot-locpot-locpt2でつくられている。
以下、順に説明する。

$n^{\rm c}+n$の第0成分から$V_{\rm xc}$の第0成分は作られる。
これは、mkpot L375あたりで呼び出されるsmvxcmでつくられる(この前に
smvxc2が呼ばれているがこれはvalence部分の寄与を取り出して評価するための
ものであり、おそらく直接にはつかわれていない---GW関連で付け加えた部分)。
通常、lfoca=1では、lfoc1=1,lfoc2=0で動いている。
smvxcmをみると、w(osmrho)にsmrho+w(ogch1)が供給されているのがわかる。
これが、$n_0+n^{\rm core}_{\rm sH}(\bfr)$であり、$n^{\rm c}+n$の第0成分
であって実meshの上で生成されている。
``C ... w(osmrho) = smrho + smoothed core from foca hankel heads''
のブロックが終了した時点でw(osmrho)には、この電子密度が入っている。これを書き出す必要がある。

{\small [ちょっと注意すべきは、w(ocgh1)である。これはその前のsmcormで生成されるが、
cgh1,cgh2の二つの部分にわけて生成されたうちの前者である。このcgh2のほうはlfoca=2の
コアがあるときのみゼロでない(なので通常のlfoca=0,1ではゼロになっている)。
このモードは本書では説明していないが
$n^{\rm core}_{\rm sH}(\bfr)$を摂動的にあつかう方法である。
lfoca=2のときも含めて対応するならcgh1+cgh2をsmrhoに足しておくのがいいか
もしれない---ただlfoca=2は小谷はつかったことがない]}

\item
$n^{\rm c}+n$の第1,2成分から$V_{\rm xc}$の第1,2成分は作られる。
これは、mkpot-locpot-locpt2で生成されている。locpt2内のコメントは
それなりにヒントになるがちょっと煩雑で混乱もある。locpt2に供給されている
のは、$n$の第1,2成分(rho1,rho2)とcoreの成分である(これをrhocと呼んでいる).
$n^{\rm c}+n$の第1,2成分はrho1+rhoc,rho2+rhochsで与えられる。
rhochsについては以下でせつめい。

すこし解説。lfoc=1のブランチに注目する。isw=1はvalenceの寄与
を計算したいときのための補助的ブランチでありisw=0をみればよい。

vxcnspがLDAのXC項を計算するルーチンであるがL813あたりの
``call dpadd(rho2(1,1,isp),rhochs,1,nr,y0/nsp)``で、
rho2=rho2+rhochsとして、vxcnspに与えている。v2が出力される.
これを実行したあとすぐに、rho2からrhochを差し引いてることに注意。

このrhochsが、$n^{\rm c}+n$の第2成分である。\req{eq:nc}の最後の項
(注意:蛇足だが、マイナス符号をとったものが第2成分)。
locpt2内でこのrhochsがどのように生成されるかをみて確認しておくとよい。

\end{itemize}


%%%%%%%%%%%%%%%%%%%%%%%%%%%%%%%%%%%%%%%%%%%%%%%%%%%%%%%%%%
\section{lmfのバンド計算のコードについて:bndfp.F}
lmfにおいてコアになっているのは、bndfp.Fである、このルーチンが一回呼ばれ
ることで、与えられた電子密度からポテンシャルをつくり、バンド計算をおこなっ
て、新たな電子密度を、返す、ということを行う(LDA+UのときはUも呼び出し
時に与えられる)。その際、電荷のmixingや、力の計算なども行うことになる。
このルーチンはlmfp.Fから呼び出される。なので、lmfp.Fのなかにiterationの
ループが存在する;do loopでなくgotoを用いており
(!! === Re-entry point for a new iteration ===)から始まっている。

構造緩和などは、lmfp.Fが担っているが、それらは将来的に
他のものとの差し替えも可能であり、bndfp.F以下が本質的な部分である。



\bibliography{lmto,gw}
\end{document}
